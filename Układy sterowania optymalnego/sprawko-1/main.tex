\documentclass[12pt,a4paper]{article}

% Polskie znaki i język
\usepackage[utf8]{inputenc}
\usepackage[T1]{fontenc}
\usepackage[polish]{babel}
\usepackage{polski}

% Matematyka
\usepackage{amsmath, amssymb, amsfonts, mathtools}
\DeclareMathSymbol{.}{\mathord}{letters}{"3B}

% Grafika i wykresy
\usepackage{graphicx}
\usepackage{float}
\usepackage{wrapfig}
\usepackage{subcaption}

% Marginesy
\usepackage{geometry}
\geometry{ a4paper, left=25mm, right=25mm, top=25mm, bottom=25mm }

% Linki i odnośniki
\usepackage{hyperref}
\hypersetup{ colorlinks=true, linkcolor=black, filecolor=magenta,      
    urlcolor=blue, pdftitle={Sprawozdanie}, }

% Kod źródłowy
\usepackage{listings}
\usepackage{xcolor}

\definecolor{codegreen}{rgb}{0,0.6,0}
\definecolor{codegray}{rgb}{0.5,0.5,0.5}
\definecolor{codepurple}{rgb}{0.58,0,0.82}
\definecolor{backcolour}{rgb}{0.95,0.95,0.92}

\usepackage{listings}
\usepackage{xcolor}

% Definicja kolorów
\definecolor{codegreen}{rgb}{0,0.6,0}
\definecolor{codegray}{rgb}{0.5,0.5,0.5}
\definecolor{codepurple}{rgb}{0.58,0,0.82}
\definecolor{backcolour}{rgb}{0.95,0.95,0.92}

% Konfiguracja stylu dla Pythona
\lstdefinestyle{mystyle}{ backgroundcolor=\color{backcolour},   
    commentstyle=\color{codegreen}, keywordstyle=\color{magenta},
    numberstyle=\tiny\color{codegray}, stringstyle=\color{codepurple},
    basicstyle=\ttfamily\footnotesize, % Czcionka maszynowa
    breakatwhitespace=false,         
    breaklines=true,                 
    captionpos=b,                    
    keepspaces=true,                 
    numbers=left,                    
    numbersep=5pt,                  
    showspaces=false,                
    showstringspaces=false, showtabs=false,                  
    tabsize=2 }

\lstset{style=mystyle}

\lstdefinestyle{mystyle}{ backgroundcolor=\color{backcolour},   
    commentstyle=\color{codegreen}, keywordstyle=\color{magenta},
    numberstyle=\tiny\color{codegray}, stringstyle=\color{codepurple},
    basicstyle=\ttfamily\footnotesize, breakatwhitespace=false,         
    breaklines=true,                 
    captionpos=b,                    
    keepspaces=true,                 
    numbers=left,                    
    numbersep=5pt,                  
    showspaces=false,                
    showstringspaces=false, showtabs=false,                  
    tabsize=2, literate={ą}{{\k{a}}}1 {ć}{{\'c}}1 {ę}{{\k{e}}}1 {ł}{{\l{}}}1 {ń}{{\'n}}1
    {ó}{{\'o}}1 {ś}{{\'s}}1 {ź}{{\'z}}1 {ż}{{\.z}}1 {Ą}{{\k{A}}}1 {Ć}{{\'C}}1
    {Ę}{{\k{E}}}1 {Ł}{{\L{}}}1 {Ń}{{\'N}}1 {Ó}{{\'O}}1 {Ś}{{\'S}}1 {Ź}{{\'Z}}1 {Ż}{{\.Z}}1
    }
\lstset{style=mystyle}

% Nagłówki i stopki
\usepackage{fancyhdr}
\pagestyle{fancy}
\fancyhf{}
\rhead{Piotr Bednarek}
\lhead{Układy sterowania optymalnego}
\cfoot{\thepage}

% Tytuł i autor
\title{Sprawozdanie\\ \large Układy sterowania optymalnego}
\author{Piotr Bednarek}
\date{\today}

\begin{document}

\maketitle
\tableofcontents
\newpage

\section{Ćwiczenia 3 i 4: Sterowalność układów liniowych}

\subsection{Równania stanu układu elektrycznego (Rys. \ref{fig:uklad2})}

\begin{figure}[H]
    \centering
    \includegraphics[width=0.5\textwidth]{img/uklad1.png}
    \caption{Układ elektryczny RC}
    \label{fig:uklad1}
\end{figure}

Przyjęto: $x_1 = u_{c_1}, \ x_2 = u_{c_2}$.

\begin{figure}[H]
    \centering
    \includegraphics[width=0.5\textwidth]{img/uklad2.png}
    \caption{Układ elektryczny RC}
    \label{fig:uklad2}
\end{figure}
Przyjęto: $x_1 = u_{c_1}, \ x_2 = u_{c_2}, \ x_3 = u_{c_3}$.




\begin{equation}
    \begin{bmatrix}
        \dot{x}_1 \\
        \dot{x}_2 \\
        \dot{x}_3
    \end{bmatrix}
    =
    \begin{bmatrix}
        -\frac{1}{R_1 C_1} & 0                  & 0                  \\
        0                  & -\frac{1}{R_2 C_2} & 0                  \\
        0                  & 0                  & -\frac{1}{R_3 C_3}
    \end{bmatrix}
    \begin{bmatrix}
        x_1 \\
        x_2 \\
        x_3
    \end{bmatrix}
    +
    \begin{bmatrix}
        \frac{1}{R_1 C_1} \\
        \frac{1}{R_2 C_2} \\
        \frac{1}{R_3 C_3}
    \end{bmatrix}
    u
\end{equation}

\subsection{Badanie sterowalności układu (Rys. \ref{fig:uklad2})}

Macierz sterowalności Kalmana $\mathcal{K}$:
\begin{equation}
    \mathcal{K} = \begin{bmatrix} B & AB & A^2B \end{bmatrix} =
    \begin{bmatrix}
        1           & -1           & 1            \\
        \frac{1}{2} & -\frac{1}{4} & \frac{1}{8}  \\
        \frac{1}{3} & -\frac{1}{9} & \frac{1}{27}
    \end{bmatrix}
\end{equation}
\\
Rząd macierzy Kalmana:
\begin{equation}
    \text{rank}(\mathcal{K}) = 3 = n
\end{equation}
Rząd macierzy Kalmana jest równy rzędowi układu, co oznacza, że układ jest
\textbf{sterowalny}.

\subsection{Analiza odpowiedzi na wymuszenie}
\begin{figure}[H]
    \centering
    \begin{subfigure}[b]{0.6\textwidth}
        \centering
        \includegraphics[width=\textwidth]{img/zmiennestanu1.png}
        \caption{Przebiegi zmiennych stanu dla układu z rysunku \ref{fig:uklad1}}
        \label{fig:przebiegi_zmiennych_stanu_uklad1}
    \end{subfigure}
    \hfill
    \begin{subfigure}[b]{0.6\textwidth}
        \centering
        \includegraphics[width=\textwidth]{img/zmiennestanu2.png}
        \caption{Przebiegi zmiennych stanu dla układu z rysunku \ref{fig:uklad2}}
        \label{fig:przebiegi_zmiennych_stanu_uklad2}
    \end{subfigure}
    \caption{Porównanie przebiegów zmiennych stanu dla układów RC z rysunków
        \ref{fig:uklad1} i \ref{fig:uklad2}}
    \label{fig:porownanie_przebiegów_zmiennych_stanu_układów}
\end{figure}

Analizując przebiegi zmiennych stanu (Rys.
\ref{fig:porownanie_przebiegów_zmiennych_stanu_układów}), można zauważyć, że dla układu z
rysunku \ref{fig:uklad1} przebiegi zmiennych stanu nakładają się na siebie, co oznacza, że
układ jest \textbf{niesterowalny}. Dla układu z rysunku \ref{fig:uklad2} przebiegi
zmiennych stanu reagują na różne wymuszenia z różnymi opóźnieniami, co oznacza, że układ
jest \textbf{sterowalny}.

\subsection{Postać sterowalna układu}
Do wyznaczenia postaci sterowalnej układu należy rozwiązać równanie:
\begin{equation}
    \phi(s) = \det(I_n s - A) = s^n + \sum_{i=0}^{n-1} a_i s^i
\end{equation}

Następne uzyskane współczynniki $a_i$ podstawiamy do macierzy sterowalnej układu:
\begin{equation}
    A_s = \begin{bmatrix}
        0      & 1      & 0      & 0        \\
        \vdots & \vdots & \ddots & \vdots   \\
        0      & 0      & 0      & 1        \\
        -a_0   & -a_1   & \dots  & -a_{n-1}
    \end{bmatrix}
\end{equation}

Macierz sterowalności układu ma postać dla układu z rysunku \ref{fig:uklad2}:
\begin{equation}
    A_s = \begin{bmatrix}
        0            & 1  & 0             \\
        0            & 0  & 1             \\
        -\frac{1}{6} & -1 & -\frac{11}{6}
    \end{bmatrix}
\end{equation}

Finalnie otrzymujemy postać sterowalną układu:
\begin{equation}
    \begin{cases}
        \dot{x_s} = \begin{bmatrix}
                        0            & 1  & 0             \\
                        0            & 0  & 1             \\
                        -\frac{1}{6} & -1 & -\frac{11}{6}
                    \end{bmatrix}
        x_s + \begin{bmatrix}
                  0 \\
                  0 \\
                  1
              \end{bmatrix}
        u_s
        \\
        \\
        y_s = \begin{bmatrix}
                  \frac{1}{2} & 2 & \frac{11}{6}
              \end{bmatrix}
        x_s
    \end{cases}
\end{equation}

\subsection{Przebiegi zmiennych stanu oraz wyjść systemów}
\begin{figure}[H]
    \centering
    \includegraphics[width=0.7\textwidth]{img/porownanie_zmiennych_stanu.png}
    \caption{Porównanie przebiegów zmiennych stanu układu w postaci sterowalnej i układu
        oryginalnego z rysunku \ref{fig:uklad2}}
    \label{fig:porownanie_zmiennych_stanu}
\end{figure}
\begin{figure}[H]
    \centering
    \includegraphics[width=0.7\textwidth]{img/wyjscia_systemow.png}
    \caption{Porównanie przebiegów wyjść układu w postaci sterowalnej i układu
        oryginalnego z rysunku \ref{fig:uklad2}}
    \label{fig:wyjscia_systemow}
\end{figure}

\newpage

\begin{itemize}
    \item \textbf{Czy obie reprezentacje opisują te same obiekty w sposób równoważny?}
          Tak, obie reprezentacje (oryginalna i sterowalna) opisują ten sam obiekt
          dynamiczny. Na wykresie \ref{fig:wyjscia_systemow} widać, że przebiegi wyjść
          układu w postaci sterowalnej i układu oryginalnego są jednakowe.

    \item \textbf{Czy przebiegi dla obu reprezentacji są jednakowe?}
          \begin{itemize}
              \item \textbf{Wyjścia $y(t)$:} Są identyczne (Rys.
                    \ref{fig:wyjscia_systemow}), ponieważ oba modele reprezentują ten sam
                    system dla zewnętrznego obserwatora.
              \item \textbf{Zmienne stanu $x(t)$:} Są różne (Rys.
                    \ref{fig:porownanie_zmiennych_stanu}). W modelu oryginalnym zmienne
                    stanu mają interpretację fizyczną (napięcia na kondensatorach),
                    natomiast w modelu sterowalnym są tylko matematycznymi zmiennymi,
                    które są kombinacjami oryginalnych zmiennych stanu.
          \end{itemize}

    \item \textbf{Znaczenie przy projektowaniu układu regulacji:} Postać sterowalna
          upraszcza projektowanie układu regulacji, ponieważ specyficzna struktura macierzy
          $A_s$ pozwala na łatwe zastosowanie metod lokowania biegunów. W modelu
          oryginalnym zmienne stanu mają interpretację fizyczną, co oznacza, że nie jest
          możliwe zastosowanie metod lokowania biegunów.
\end{itemize}

\subsection{Lokowanie biegunów}
\textbf{Przedstawienie metody lokowania biegunów:}

Wyznaczenie pierwiastków układu charakterystycznego:
\begin{equation}
    \det(sI - (A - BK)) = 0
\end{equation}

Macierz wzmocnień $K$ dla postaci sterowalnej:
\begin{equation}
    K = \begin{bmatrix}
        9.8333 & 16.0 & 6.1667
    \end{bmatrix}
\end{equation}

Każdy element wektora wzmocnień $K$ bezpośrednio wpływa na konkretny współczynnik równania
charakterystycznego, co oznacza, że każdy element wektora $K$ wpływa na konkretny biegun
układu.

Bieguny układu sterowalnego:
\begin{equation}
    \begin{cases}
        s_1 = -1 \\
        s_2 = -2 \\
        s_3 = -5
    \end{cases}
\end{equation}

Wymuszenie zdefiniowano jako $u = -K x$, co oznacza, że bez zewnetrznych wymuszeń układ
będzie dążył do stanu równowagi. Aby zobaczyć działanie regulatora, należy zastosować
\textbf{niezerowe} warunki początkowe.

\begin{figure}[H]
    \centering
    \includegraphics[width=0.7\textwidth]{img/symulacja-ukladu-zamknietego.png}
    \caption{Odpowiedź układu zamkniętego na warunki początkowe $\dot{x}(0) =
            \begin{bmatrix} 1 & 1 & 1 \end{bmatrix}^T$}
    \label{fig:symulacja_ukladu_zamknietego}
\end{figure}

\section{Ćwiczenie 5: Podstawy optymalizacji matematycznej}


\textbf{Przedstawiono problem optymalizacji dynamicznej:}
\begin{equation}
    J = \int_{0}^{1} 24x(t)t + 2\dot{x}(t)^2 - 4t \, dt \rightarrow min
\end{equation}

z ograniczeniami:
\begin{equation}
    x(0) = 1, \quad x(1) = 3
\end{equation}

Minimalna wartość całki uzyskana numerycznie: $J_{min} = 39.24$

\subsection{Porównanie rozwiązania analitycznego i numerycznego}

\begin{figure}[H]
    \centering
    \includegraphics[width=0.6\textwidth]{img/porownanie-numeryczne-analityczne.png}
    \caption{Porównanie rozwiązania analitycznego i numerycznego}
    \label{fig:porownanie-numeryczne-analityczne}
\end{figure}

\begin{itemize}
    \item \textbf{Czy rozwiązanie pokrywa się z tym wyznaczonym analitycznie?} \\
          Nie, ponieważ rozwiązanie numeryczne jest nieprecyzyjne.
    \item \textbf{Jaka metoda optymalizacji jest wykorzystywana w pakiecie
              \textbf{\textit{gekko}}?} \\
          Tryb \textbf{IMODE = 6} wykorzystuje metodę \textit{Dynamicznej Optymalizacji} -
          minimalizacja złożonej funkcji celu.

\end{itemize}

\newpage

\subsection{Kod programu do rozwiązania zadania optymalizacji}
\begin{lstlisting}[language=Python, caption=Rozwiązanie w bibliotece GEKKO]
import numpy as np 
from gekko import GEKKO 
import matplotlib.pyplot as plt

# Parametry 
n = 101

# Model 
model = GEKKO() 
model.options.IMODE = 6

# Parametry 
model.time = np.linspace(0, 1, n) 
t = model.Param(value=model.time)

# Ograniczenia 
x = model.Var() 
model.fix_initial(x, 1) 
model.fix_final(x, 3)

# Inicjalizacja całki 
J = model.Var(value=0) 
model.fix_initial(J, 0)

# Równanie różniczkowe 
model.Equation(J.dt() == 24 * x * t + 2 * x.dt()**2 - 4 * t)

# Cel 
final_cost = model.FV() 
final_cost.STATUS = 1 
model.Connection(final_cost, J, pos2='end') 
model.Obj(final_cost)

# Konfiguracja solvera 
model.options.SOLVER = 3

# Rozwiązanie 
model.solve(disp=False)

print(f"\nMinimalna wartość całki (GEKKO): {final_cost.value[0]:.2f}")
\end{lstlisting}

\section{Ćwiczenie 6: Dobór optymalnych nastaw regulatora PID}
\subsection{Przykładowy przebieg odpowiedzi systemu}

\begin{figure}[H]
    \centering
    \includegraphics[width=0.6\textwidth]{img/odpowiedz-ukladu-pid.png}
    \caption{Przykładowy przebieg odpowiedzi systemu}
    \label{fig:przykladowy-przebieg-odpowiedzi-systemu}
\end{figure}

Wybrane nastawy regulatora PID:
\begin{equation}
    \begin{cases}
        K_p = 5 \\
        K_i = 8 \\
        K_d = 2
    \end{cases}
\end{equation}

\begin{itemize}
    \item \textbf{Jak zwiększanie/zmniejszanie wartości każdego z parametrów regulatora
              PID wpływają na odpowiedź skokową układu?}
          \begin{enumerate}
              \item \textbf{$K_p$ - wzmocnienie proporcjonalne} \begin{itemize}
                        \item \textbf{Zwiększanie}: Zwiększa szybkość reakcji układu, ale
                              zwiększa też przeregulowanie. Zbyt duże $K_p$ może powodować
                              oscylacje.
                        \item \textbf{Zmniejszanie}: Układ reaguje wolniej, ale jest
                              stabilniejszy.
                    \end{itemize}

              \item \textbf{$K_i$ - wzmocnienie całkujące} \begin{itemize}
                        \item \textbf{Zwiększanie}: Szybsza eliminacja uchybu kosztem
                              ryzyka wprowadzenia niestabilności przy zbyt dużej wartości.
                        \item \textbf{Zmniejszanie}: Wolniejsza redukcja uchybu, ale
                              mniejsze oscylacje.
                    \end{itemize}

              \item \textbf{$K_d$ - wzmocnienie różniczkujące} \begin{itemize}
                        \item \textbf{Zwiększanie}: Tłumi oscylacje wprowadzane przez człon
                              proporcjonalny. Skraca czas ustalania.
                        \item \textbf{Zmniejszanie}: Słabsze tłumienie oscylacji, układ
                              bardziej podatny na przeregulowania wprowadzone przez człon
                              proporcjonalny.
                    \end{itemize}
          \end{enumerate}
    \item \textbf{Czy możliwe jest uzyskanie zerowego uchybu ustalonego dla regulatora
              typu P?} \\ Nie, ponieważ układ RLC wymaga utrzymania niezerowego sygnału
          wejściowego. W regulatorze typu P sygnał sterujący jest proporcjonalny do
          uchybu. W przypadku, gdzie uchyb spada do zera, sygnał sterujący również
          spada do zera, co powoduje, że układ RLC zaczyna się poruszać w drugą
          stronę.
    \item \textbf{Jakie są najmniejsze wartości nastaw regulatora PI gwarantujące
              zerowanie uchybu ustalonego?} \\ Teoretycznie każda wartość $K_i >0$. Człon
          całkujący wprowadza do układu biegun w zerze, co matematycznie wymusza
          redukcję uchybu do zera w stanie ustalonym, niezależnie od wielkości
          wzmocnienia.
    \item \textbf{Czy możliwe jest uzyskanie czasu ustalania mniejszego niż $2s$? Dla
              jakich nastaw i jakiego typu regulatora?} \\ Tak, możliwe jest uzyskanie
          czasu ustalania mniejszego niż $2s$ dla regulatora typu PID. \\
\end{itemize}

\subsection{Dobór nastaw metodą Ziegler-Nicholsa}
Wartość wzmocnienia granicznego dobrana eksperymentalnie wynosi $k_u = 240$. Okres drgań
odpowiedzi układu $T_u = 1.8220 s$.

Dla parametrów krytycznych $K_u=30.5$ oraz $T_u=3.14$ wyznaczono nastawy dla regulatorów
P, PI, PD oraz PID korzystając z tabeli Zieglera-Nicholsa:

\begin{table}[H]
    \centering
    \begin{tabular}{|c|c|c|c|}
        \hline
        Regulator & $K_p$  & $K_i$  & $K_d$ \\
        \hline
        P         & 15,250 & ---    & ---   \\
        PI        & 13,725 & 5,245  & ---   \\
        PD        & 24,400 & ---    & 9,577 \\
        PID       & 18,300 & 11,656 & 7,183 \\
        \hline
    \end{tabular}
    \caption{Wyliczone nastawy regulatorów}
    \label{tab:nastawy_zn_nowe}
\end{table}

\begin{figure}[H]
    \centering
    \includegraphics[width=0.6\textwidth]{img/pid-ziegler-nichols.png}
    \caption{Porównanie odpowiedzi systemu dla regulatorów P, PI, PD i PID dobranych metodą Ziegler-Nicholsa}
    \label{fig:porownanie-odpowiedzi-systemu}
\end{figure}
\subsection{Całkowe wskaźniki jakości}

W tabeli \ref{tab:wskazniki_jakosci} przedstawiono wartości wskaźników jakości dla różnych
nastaw regulatora. Zestawienie obejmuje błąd kwadratowy (ISE), błąd z wagą czasu (ITAE)
oraz koszt całkowity $I_{OPT}$ obliczony dla dwóch scenariuszy: drogiego sterowania
($r=1$) oraz taniego sterowania ($r=0.01$).

\begin{table}[H]
    \centering
    \setlength{\tabcolsep}{5pt} % Zmniejszenie odstępów w kolumnach dla lepszego
    \begin{tabular}{|l|c|c|c|c|}
        \hline
        \textbf{Nazwa nastaw} & \textbf{ISE}              & \textbf{ITAE}   & \textbf{Koszt ($r=1$)}
                              & \textbf{Koszt ($r=0.01$)}                                            \\
        \hline
        Ziegler-Nichols       & 1,1117                    & 3,3700          & 4142,4496
                              & 42,5251                                                              \\
        \hline
        Min. ISE / Szybki     & \textbf{0,7117}           & \textbf{2,2469} & 5776,9423
                              & 58,4740                                                              \\
        \hline
        Min. IOPT / Oszczędny & 4,1292                    & 26,0145         & \textbf{1067,9123}
                              & \textbf{14,7670}                                                     \\
        \hline
    \end{tabular}
    \caption{Porównanie wskaźników całkowych dla różnych nastaw}
    \label{tab:wskazniki_jakosci}
\end{table}

\begin{figure}[H]
    \centering
    \includegraphics[width=0.5\textwidth]{img/wskazniki-jakosci.png}
    \caption{Porównanie wskaźników całkowych dla różnych nastaw}
    \label{fig:wskazniki_jakosci}
\end{figure}

\section{Ćwiczenia 7-8: Regulator optymalny LQR}
\subsection{Odpowiedź skokowa i regulator LQR}

\begin{figure}[H]
    \centering
    \includegraphics[width=0.5\textwidth]{img/uklad-rlc.png}
    \caption{Badany układ RLC}
    \label{fig:uklad-rlc}
\end{figure}

Przyjęto parametry układu: $R = 0.5\,\Omega$, $C = 0.5\,\text{F}$ oraz $L = 0.2\,\text{H}$.

\begin{figure}[H]
    \centering
    \includegraphics[width=0.75\textwidth]{img/wymuszenie-skokowe.png}
    \caption{Odpowiedź układu z rysunku \ref{fig:uklad-rlc} na wymuszenie skokowe}
    \label{fig:odpowiedz-skokowa}
\end{figure}

Wyznaczone wartości macierzy wzmocnień $K$:
\begin{equation}
    K = \begin{bmatrix}
        0.0990 & 0.6615
    \end{bmatrix}
\end{equation}

\newpage

Odpowiedź układu zamkniętego z regulatorem LQR na wymuszenie skokowe:

\begin{figure}[H]
    \centering
    \includegraphics[width=0.75\textwidth]{img/odpowiedz-lqr.png}
    \caption{Odpowiedź układu z rysunku \ref{fig:uklad-rlc} z regulatorem LQR na wymuszenie skokowe}
    \label{fig:odpowiedz-lqr}
\end{figure}

Wpływ zmian wartości macierzy $Q$ oraz $R$ na odpowiedź układu:

\begin{figure}[H]
    \centering
    \includegraphics[width=0.75\textwidth]{img/rozne-macierze.png}
    \caption{Porównanie odpowiedzi układu dla różnych wartości macierzy wag regulatora LQR}
    \label{fig:porownanie-odpowiedzi-lqr-macierze}
\end{figure}

\newpage

Z przeprowadzonej analizy przebiegów na rysunku
\ref{fig:porownanie-odpowiedzi-lqr-macierze} można zauważyć, że nadanie bardzo wysokiego
priorytetu minimalizacji błędu (duża wartość macierzy $Q$) powoduje bardzo szybką
stabilizację systemu kosztem agresywnego sterowania. Z kolei duży nacisk na oszczędzanie
energii (duża wartość macierzy $R$) sprawia, że sterowanie staje się "tanie" i łagodne, co
minimalizuje koszty eksploatacji, lecz znacznie wydłuża czas dochodzenia do
równowagi. Ostatecznie, dobór optymalnych nastaw regulatora LQR jest zawsze kompromisem i
wymaga ustalenia priorytetów projektowych: czy kluczowa jest maksymalna dynamika i
precyzja, czy też minimalizacja zużycia energii i ochrona elementów wykonawczych.

\subsection{Porównanie wyników}

\begin{figure}[H]
    \centering
    \includegraphics[width=0.75\textwidth]{img/macierz-p.png}
    \caption{Przebieg elementów macierzy $P$}
    \label{fig:macierz-p}
\end{figure}

\subsubsection{Porównanie LQR ze skończonym i nieskończonym horyzontem czasowym}
Celem zadania jest analiza porównawcza jakości sterowania regulatora LQR działającego w
skończonym horyzoncie czasowym (ang. \textit{Finite Horizon}) względem regulatora z
horyzontem nieskończonym (ang. \textit{Infinite Horizon}), który przyjęto jako wzorzec
odniesienia. Zbadano wpływ długości horyzontu sterowania $t_1$ oraz wagi kary końcowej $S$
na przebieg zmiennych stanu. Przyjęto następujące zestawy parametrów:
\begin{itemize}
    \item Czas symulacji (horyzont): $t_1 \in \{1, 2, 5\}$ s,
    \item Macierz kary końcowej: $S \in \{I_2, 100I_2\}$.
\end{itemize}

Poniżej przedstawiono zestawienie odpowiedzi układu dla wszystkich kombinacji parametrów,
ze szczególnym uwzględnieniem zdolności regulatora do poprawnej stabilizacji układu w
zadanym czasie.

\begin{figure}[H]
    \centering
    \includegraphics[width=\textwidth]{img/porownanie-horyzontow.png}
    \caption{Porównanie odpowiedzi układu dla różnych horyzontów czasowych oraz różnych zestawów parametrów}
    \label{fig:porownanie-horyzontow}
\end{figure}

Zadanie stabilizacji układu zostało poprawnie zrealizowane w dwóch scenariuszach:

\begin{enumerate}
    \item \textbf{Skończony horyzont czasowy}: \begin{itemize}
              \item $t_1 = 1$ s, $S = 100I_2$.
              \item $t_1 = 2$ s, $S = 100I_2$.
              \item $t_1 = 5$ s, $S = I_2$.
              \item $t_1 = 5$ s, $S = 100I_2$.
          \end{itemize}

    \item \textbf{Nieskończony horyzont czasowy}: \begin{itemize}
              \item $t_1 = 5$ s, $S = I_2$.
              \item $t_1 = 5$ s, $S = 100I_2$.
          \end{itemize}
\end{enumerate}

Powyższe zestawienie potwierdza, że skuteczna stabilizacja w skończonym horyzoncie
czasowym jest możliwa w dwóch przypadkach: gdy czas regulacji jest wystarczająco długi
($t_1=5$ s), co pozwala na naturalne wygaszenie stanu przejściowego niezależnie od doboru
macierzy $S$, lub gdy narzucona jest bardzo wysoka kara końcowa ($S=100I_2$), która
wymusza na regulatorze agresywne działanie w celu wyzerowania uchybu przed upływem
krótkiego czasu ($t_1=1, 2$ s).

\subsection{Stabilizacja w punkcie innym niż zero}

W ramach zadania zmodyfikowano model symulacyjny, aby umożliwić stabilizację układu w
zadanym, niezerowym punkcie równowagi. Sygnał sterujący ma postać:

\begin{equation}
    u(t) = -u_e + \frac{1}{C}q_d
\end{equation}

Przeprowadzono symulacje odpowiedzi układu zamkniętego dla trzech różnych wartości
zadanych ładunku $q_d \in \{1, 2, 5\}$. Poniżej przedstawiono kod wykorzystany do
rozwiązania tego zadania.

\begin{lstlisting}[basicstyle=\scriptsize, language=Python, caption=Kod do rozwiązania zadania stabilizacji w punkcie innym niż zero]
import numpy as np
from scipy.integrate import odeint
from scipy.linalg import solve_continuous_are
import matplotlib.pyplot as plt

R_val = 0.5
L_val = 0.5
C_val = 0.2

A = np.array([[0, 1], [-1 / (L_val * C_val), -R_val / L_val]])
B = np.array([[0], [1 / L_val]])
Q = np.eye(2)
R = np.array([[1]])
P = solve_continuous_are(A, B, Q, R)
K = np.linalg.inv(R) @ B.T @ P

def model(x, t, K, q_t):
    x_t = np.array([q_t, 0])
    e = x_t - x
    u_c = q_t / C_val
    u = K @ e + u_c
    dxdt = A @ x + B.flatten() * u[0]
    return dxdt

t = np.linspace(0, 10, 1000)
x0 = np.array([0, 0])

sol1 = odeint(model, x0, t, args=(K, 1))
sol2 = odeint(model, x0, t, args=(K, 2))
sol3 = odeint(model, x0, t, args=(K, 5))

plt.figure(figsize=(12, 7))
plt.plot(t, sol1[:, 0], 'g-', linewidth=2.5, label=r'Wartość zadana $q_d = 1$')
plt.axhline(1, color='g', linestyle=':', alpha=0.5)
plt.plot(t, sol2[:, 0], 'b--', linewidth=2.5, label=r'Wartość zadana $q_d = 2$')
plt.axhline(2, color='b', linestyle=':', alpha=0.5)
plt.plot(t, sol3[:, 0], 'r-.', linewidth=2.5, label=r'Wartość zadana $q_d = 5$')
plt.axhline(5, color='r', linestyle=':', alpha=0.5)
plt.title('Realizacja zadania śledzenia wartości zadanej', fontsize=20, fontweight='bold', pad=20)
plt.xlabel('Czas [s]', fontsize=16)
plt.ylabel('Ładunek $x_1(t)$', fontsize=16)
plt.legend(fontsize=14)
plt.grid(True, linestyle='--', alpha=0.7)
plt.tight_layout()
plt.show()
\end{lstlisting}

\newpage

Wyniki symulacji przedstawiono na rysunku poniżej.

\begin{figure}[H]
    \centering
    \includegraphics[width=0.75\textwidth]{img/sledzenie-wartosci-zadanej.png}
    \caption{Porównanie odpowiedzi układu dla różnych horyzontów czasowych oraz różnych zestawów parametrów}
    \label{fig:sledzenie-wartosci-zadanej}
\end{figure}

Analiza wyników potwierdza, że regulator LQR (pierwotnie zaprojektowany dla problemu
stabilizacji w zerze) może być skutecznie wykorzystany do realizacji zadania śledzenia
stałej wartości zadanej. Możliwość ta wynika z dwóch czynników:

\begin{itemize}
    \item \textbf{Przesunięcie układu współrzędnych:} Poprzez wprowadzenie wektora uchybu
          $e = x_d - x$, dynamika błędu sprowadza się do problemu stabilizacji w zerze. Macierz
          wzmocnień $K$ obliczona dla układu nominalnego zapewnia identyczną jakość tłumienia
          uchybu, niezależnie od wartości zadanej.
    \item \textbf{Kompensacja (Feedforward):} Warunkiem koniecznym utrzymania niezerowego
          stanu jest podanie stałego sygnału sterującego $u_c$, który równoważy dynamikę układu
          w punkcie równowagi (w tym przypadku równoważy napięcie na naładowanym kondensatorze).
          Bez tego składnika układ dążyłby do rozładowania (powrotu do zera).
\end{itemize}

Warto zauważyć, że ze względu na liniowość obiektu, charakter odpowiedzi dynamicznej (czas
narastania, przeregulowanie) jest identyczny dla każdej amplitudy wymuszenia $q_d$,
zmienia się jedynie skala sygnałów.



\end{document}
