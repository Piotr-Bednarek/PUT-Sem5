\documentclass[11pt]{article}

\pdfobjcompresslevel=3    % compress PDF objects
\pdfcompresslevel=9       % compress streams more (0..9)
\pdfminorversion=4


\usepackage[utf8]{inputenc} % remove if using XeLaTeX or LuaLaTeX
\usepackage[T1]{fontenc}

% Layout & graphics
\usepackage[a4paper, total={7.5in, 9in}]{geometry}
\usepackage{graphicx}              % images
\usepackage[dvipsnames,table]{xcolor} % single xcolor invocation (keeps table option)

% Math & symbols
\usepackage{amsmath}
\usepackage{amssymb}
\usepackage{gensymb}

% Typography & layout helpers
\usepackage{microtype}
\usepackage{float}                 % [H] placement
\usepackage{caption}
\usepackage{subcaption}            % modern subfigure support (do NOT load subfig)
\usepackage{multirow}
\usepackage{titlesec}

\usepackage{lmodern}
\usepackage{microtype}

\usepackage{pgfplots}
\usepackage{siunitx}
\pgfplotsset{compat=1.18}

\definecolor{xppblue}{RGB}{37, 150, 190}

\begin{document}

\begin{table}[h]
    \centering
    \scalebox{1.5}{
        \begin{tabular}{|c|p{3cm}|l|l|cl|}
            \hline
            \multicolumn{4}{|c|}{SPRAWOZDANIE Z LABORATORIUM}                                  & \multicolumn{2}{c|}{\multirow{2}{*}{\begin{tabular}[c]{@{}c@{}}rok akademicki:\\ 2025/26\end{tabular}}}                                                          \\ \cline{1-4}
            \multicolumn{4}{|c|}{\textbf{Układy elektroniki użytkowej}}                        & \multicolumn{2}{c|}{}                                                                                                                                            \\ \hline
            \multicolumn{4}{|c|}{\multirow{2}{*}{\textit{Wzmacniacz z regulacją wzmocnienia}}} & \multicolumn{2}{c|}{\multirow{2}{*}{czwartek 8:00}}                                                                                                              \\
            \multicolumn{4}{|c|}{}                                                             & \multicolumn{2}{c|}{}                                                                                                                                            \\ \hline
            \multicolumn{1}{|c|}{WARiE, AiR, sem 5}                                            & \multicolumn{3}{p{5cm}|}{\begin{tabular}[c]{@{}l@{}}1. Piotr Cybal\\ \underline{2. Piotr Bednarek}\end{tabular}} & \multicolumn{2}{l|}{\multirow{2}{*}{Punkty:}} \\ \cline{1-1}
            \multicolumn{1}{|c|}{15.10.2025}                                                   & \multicolumn{3}{c|}{}                                                                                            & \multicolumn{2}{l|}{}                         \\ \hline
        \end{tabular}
    }
\end{table}

\vspace{1\baselineskip}

\section{Sprzęt}
\begin{itemize}
    \item Numer stanowiska: 8
    \item Numer Elvisa: A
    \item Elementy elektroniczne: 1
\end{itemize}

\section{Ćwiczenie}

Na początku ćwiczenia złożyliśmy układ wzmacniacza operacyjnego z regulacją wzmocnienia za
pomocą dobrania odpowiednich rezystorów. Następnie wykonywaliśmy pomiary wzmocnienia
układu przy różnych częstotliwościach generowanej funkcji oraz porównywaliśmy z wartościami teoretycznymi uzyskanymi z odpowiednich wzorów.


\section{Wzmacniacz nieodwracający}

\subsection{Układ wzmacniacza}
\begin{figure}[H]
    \centering
    \includegraphics[width=0.4\linewidth]{img/uklad1.png}
    \caption{Układ wzmacniacza nieodwracającego}
    \label{fig:uklad}
\end{figure}

\subsection{Pomiary}
Do układu dobraliśmy 3 zestawy rezystorów:
\begin{itemize}
    \item $R_1 = 100k\Omega$, $R_2 = 10k\Omega$,
    \item $R_1 = 100k\Omega$, $R_2 = 5.1k\Omega$,
    \item $R_1 = 100k\Omega$, $R_2 = 2k\Omega$.\
\end{itemize}

Co ze wzoru:
\begin{equation}
    k_u =\frac{R_1}{R_2} + 1
\end{equation}
Daje nam odpowiednio wzmocnienia:
\begin{itemize}
    \item $k_u = 11$,
    \item $k_u = 20.6$,
    \item $k_u = 51$.
\end{itemize}

\begin{table}[H]
    \centering
    \begin{tabular}{|c|c|c|c|c|c|c|c|c|c|c|}
        \hline
        \textbf{R1}  & \textbf{R2}  & \textbf{100 Hz} & \textbf{500 Hz} & \textbf{1 kHz} & \textbf{3 kHz} & \textbf{5 kHz} & \textbf{8 kHz} & \textbf{10 kHz} & \textbf{15 kHz} & \textbf{20 kHz} \\ \hline
        $100k\Omega$ & $10k\Omega$  & 10,9159         & 10,9142         & 10,9031        & 10,9006        & 10,8941        & 10,8827        & 10,8907         & 10,8628         & 10,8489         \\ \hline
        $100k\Omega$ & $5.1k\Omega$ & 20,7601         & 20,7627         & 20,7549        & 20,7395        & 20,7314        & 20,6962        & 20,6908         & 20,588          & 20,4678         \\ \hline
        $100k\Omega$ & $2k\Omega$   & 50,7872         & 50,7592         & 50,7613        & 50,658         & 50,4537        & 50,1147        & 49,7709         & 48,8505         & 47,5561         \\ \hline
    \end{tabular}
    \caption{Pomiary wzmocnienia układu przy różnych częstotliwościach}
\end{table}

\begin{figure}[h!]
    \centering
    \begin{tikzpicture}
        \begin{axis}[
            width=14cm,
            height=8cm,
            xlabel={Częstotliwość [Hz]},
            ylabel={Wzmocnienie napięciowe $k_u$},
            title={Zależność wzmocnienia od częstotliwości},
            grid=both,
            xmode=log,
            log basis x=10,
            legend style={
                    at={(0.5,-0.2)},
                    anchor=north,
                    legend columns=-1,
                    /tikz/every even column/.append style={column sep=0.5cm}
                },
            ymin=0,
            ymajorgrids=true,
            xtick={100,500,1000,3000,5000,8000,10000,15000,20000},
            xticklabels={100,500,1k,3k,5k,8k,10k,15k,20k},
            ]

            % --- R2 = 10kΩ ---
            \addplot[
                thick,
                color=blue,
                mark=*,
            ] coordinates {
                    (100,10.9159)
                    (500,10.9142)
                    (1000,10.9031)
                    (3000,10.9006)
                    (5000,10.8941)
                    (8000,10.8827)
                    (10000,10.8907)
                    (15000,10.8628)
                    (20000,10.8489)
                };
            \addlegendentry{$R_2 = 10\,\mathrm{k}\Omega$}

            % --- R2 = 5.1kΩ ---
            \addplot[
                thick,
                color=red,
                mark=square*,
            ] coordinates {
                    (100,20.7601)
                    (500,20.7627)
                    (1000,20.7549)
                    (3000,20.7395)
                    (5000,20.7314)
                    (8000,20.6962)
                    (10000,20.6908)
                    (15000,20.588)
                    (20000,20.4678)
                };
            \addlegendentry{$R_2 = 5.1\,\mathrm{k}\Omega$}

            % --- R2 = 2kΩ ---
            \addplot[
                thick,
                color=green!60!black,
                mark=triangle*,
            ] coordinates {
                    (100,50.7872)
                    (500,50.7592)
                    (1000,50.7613)
                    (3000,50.658)
                    (5000,50.4537)
                    (8000,50.1147)
                    (10000,49.7709)
                    (15000,48.8505)
                    (20000,47.5561)
                };
            \addlegendentry{$R_2 = 2\,\mathrm{k}\Omega$}

        \end{axis}
    \end{tikzpicture}
    \caption{Zależność wzmocnienia napięciowego od częstotliwości dla układu nieodwracającego}
\end{figure}

\section{Wzmacniacz odwracający}
\subsection{Układ wzmacniacza}
\begin{figure}[H]
    \centering
    \includegraphics[width=0.4\linewidth]{img/uklad2.png}
    \caption{Układ wzmacniacza odwracającego}
    \label{fig:uklad2}
\end{figure}

\subsection{Pomiary}
Do układu dobraliśmy 3 zestawy rezystorów:
\begin{itemize}
    \item $R_1 = 1k\Omega$, $R_2 = 10k\Omega$,
    \item $R_1 = 10k\Omega$, $R_2 = 5k\Omega$,
    \item $R_1 = 100k\Omega$, $R_2 = 10k\Omega$.\
\end{itemize}

Co ze wzoru:
\begin{equation}
    k_u = -\frac{R_1}{R_2}
\end{equation}
Daje nam odpowiednio wzmocnienia:
\begin{itemize}
    \item $k_u = -0.1$,
    \item $k_u = -2$,
    \item $k_u = -10$.
\end{itemize}


\begin{table}[H]
    \centering
    \begin{tabular}{|c|c|c|c|c|c|c|c|c|c|c|}
        \hline
        \textbf{R1}  & \textbf{R2} & \textbf{100 Hz} & \textbf{500 Hz} & \textbf{1 kHz} & \textbf{3 kHz} & \textbf{5 kHz} & \textbf{8 kHz} & \textbf{10 kHz} & \textbf{15 kHz} & \textbf{20 kHz} \\ \hline
        $1k\Omega$   & $10k\Omega$ & -0,100865       & -0,100872       & -0,10086       & -0,10088       & -0,10086       & -0,10086       & -0,10079        & -0,10088        & -0,10088        \\ \hline
        $10k\Omega$  & $5k\Omega$  & -1,97427        & -1,9743         & -1,97436       & -1,97454       & -1,97443       & -1,97446       & -1,97453        & -1,9742         & -1,97404        \\ \hline
        $100k\Omega$ & $10k\Omega$ & -10,0458        & -10,046         & -10,0445       & -10,0431       & -10,0414       & -10,0376       & -10,0319        & -10,0152        & -9,99514        \\ \hline
    \end{tabular}
    \caption{Pomiary wzmocnienia układu przy różnych częstotliwościach}
\end{table}

\begin{figure}[h!]
    \centering
    \begin{tikzpicture}
        \begin{axis}[
            width=14cm,
            height=8cm,
            xlabel={Częstotliwość [Hz]},
            ylabel={Wzmocnienie napięciowe $A_u$},
            title={Zależność wzmocnienia od częstotliwości},
            grid=both,
            xmode=log,
            log basis x=10,
            ymin=-11,
            ymax=1,
            ymajorgrids=true,
            xtick={100,500,1000,3000,5000,8000,10000,15000,20000},
            xticklabels={100,500,1k,3k,5k,8k,10k,15k,20k},
            legend style={
                    at={(0.5,-0.25)},
                    anchor=north,
                    legend columns=-1,
                    /tikz/every even column/.append style={column sep=0.6cm}
                },
            ]

            % --- R1 = 1kΩ, R2 = 10kΩ ---
            \addplot[
                thick,
                color=blue,
                mark=*,
            ] coordinates {
                    (100,-0.100865)
                    (500,-0.100872)
                    (1000,-0.10086)
                    (3000,-0.10088)
                    (5000,-0.10086)
                    (8000,-0.10086)
                    (10000,-0.10079)
                    (15000,-0.10088)
                    (20000,-0.10088)
                };
            \addlegendentry{$R_1 = 1\,\mathrm{k}\Omega$, $R_2 = 10\,\mathrm{k}\Omega$}

            % --- R1 = 10kΩ, R2 = 5kΩ ---
            \addplot[
                thick,
                color=red,
                mark=square*,
            ] coordinates {
                    (100,-1.97427)
                    (500,-1.9743)
                    (1000,-1.97436)
                    (3000,-1.97454)
                    (5000,-1.97443)
                    (8000,-1.97446)
                    (10000,-1.97453)
                    (15000,-1.9742)
                    (20000,-1.97404)
                };
            \addlegendentry{$R_1 = 10\,\mathrm{k}\Omega$, $R_2 = 5\,\mathrm{k}\Omega$}

            % --- R1 = 100kΩ, R2 = 10kΩ ---
            \addplot[
                thick,
                color=green!60!black,
                mark=triangle*,
            ] coordinates {
                    (100,-10.0458)
                    (500,-10.046)
                    (1000,-10.0445)
                    (3000,-10.0431)
                    (5000,-10.0414)
                    (8000,-10.0376)
                    (10000,-10.0319)
                    (15000,-10.0152)
                    (20000,-9.99514)
                };
            \addlegendentry{$R_1 = 100\,\mathrm{k}\Omega$, $R_2 = 10\,\mathrm{k}\Omega$}

        \end{axis}
    \end{tikzpicture}
    \caption{Zależność wzmocnienia napięciowego od częstotliwości dla układu odwracającego}
\end{figure}

\section{Wnioski}

\subsection{Wnioski na podstawie wykresów wzmocnień}
Możemy zauważyć stabilność wzmocnienia dla większości badanych częstotliwości dla obu
układów. Wykresy są niemal idealnie płaskie, co oznacza, że wzmocnienie nie zmienia się
znacząco wraz ze zmianą częstotliwości sygnału wejściowego. \\
Dla układu nieodwracającego wzmocnienie zaczyna lekko spadać przy wyższych częstotliwościach (od około 8 kHz),
szczególnie dla największego wzmocnienia $k_u=51$.

\subsection{Ogólne}
Ćwiczenie pozwoliło nam precyzyjnie kontrolować wzmocnienie sygnału odpowiednimi
rezystorami, a następnie obserwować jego rzeczywistą wartość. Zaobserwowaliśmy również
ograniczone pasmo przenoszenia wzmacniacza operacyjnego. W praktyce oznacza to, że przy
bardzo wysokich częstotliwościach wzmocnienie może ulec znacznemu obniżeniu.

\end{document}
