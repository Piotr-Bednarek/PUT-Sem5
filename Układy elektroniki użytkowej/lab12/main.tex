\documentclass[11pt]{article}

\pdfobjcompresslevel=3    % compress PDF objects
\pdfcompresslevel=9       % compress streams more (0..9)
\pdfminorversion=4


\usepackage[utf8]{inputenc} % remove if using XeLaTeX or LuaLaTeX
\usepackage[T1]{fontenc}

% Layout & graphics
\usepackage[a4paper, total={7.5in, 9in}]{geometry}
\usepackage{graphicx}              % images
\usepackage[dvipsnames,table]{xcolor} % single xcolor invocation (keeps table option)

% Math & symbols
\usepackage{amsmath}
\usepackage{amssymb}
\usepackage{gensymb}

% Typography & layout helpers

\usepackage{polski}[babel]

\usepackage{float}                 %  placement
\usepackage{caption}
\usepackage{subcaption}            % modern subfigure support (do NOT load subfig)
\usepackage{multirow}
\usepackage{titlesec}

\usepackage{lmodern}
\usepackage{microtype}

\usepackage{pgfplots}
\usepackage{booktabs}
\usepackage{siunitx}

\pgfplotsset{compat=1.18, /pgf/number format/use comma}
\definecolor{xppblue}{RGB}{37, 150, 190}


\begin{document}

\begin{table}
    \centering
    \scalebox{1.5}{
        \begin{tabular}{|c|p{3cm}|l|l|cl|}
            \hline
            \multicolumn{4}{|c|}{SPRAWOZDANIE Z LABORATORIUM}                              & \multicolumn{2}{c|}{\multirow{2}{*}{\begin{tabular}[c]{@{}c@{}}rok akademicki:\\ 2025/26\end{tabular}}}
            \\ \cline{1-4} \multicolumn{4}{|c|}{\textbf{Układy elektroniki użytkowej}}                    & \multicolumn{2}{c|}{}
            \\ \hline
            \multicolumn{4}{|c|}{\multirow{2}{*}{\textit{Aplikacje układu czasowego 555}}} & \multicolumn{2}{c|}{\multirow{2}{*}{czwartek 8:00}}
            \\
            \multicolumn{4}{|c|}{}                                                         & \multicolumn{2}{c|}{}
            \\ \hline
            \multicolumn{1}{|c|}{WARiE, AiR, sem 5}                                        & \multicolumn{3}{p{5cm}|}{\begin{tabular}[c]{@{}l@{}}1. Piotr Cybal\\ \underline{2. Piotr Bednarek}\end{tabular}} &
            \multicolumn{2}{l|}{\multirow{2}{*}{Punkty:}}                                                                                                                                                       \\ \cline{1-1} \multicolumn{1}{|c|}{18.12.2025}                                               & \multicolumn{3}{c|}{}
                                                                                           & \multicolumn{2}{l|}{}                                                                                              \\ \hline
        \end{tabular}
    }
\end{table}

\vspace{1\baselineskip}

\section*{Sprzęt}
\begin{itemize}
    \item Numer stanowiska: 8
    \item Numer Elvisa: G
    \item Numer skrzynki z elementami: 4
\end{itemize}

% \section{Cel ćwiczenia}


\section{Badanie generatora sterowanego napięciem (VCO)}
\subsection{Potencjometr w położeniu środkowym}

Pomiary przy regulacji napięciem sterującym od 0 do 5V:

\begin{center}
    \begin{tabular}{lcc}
        \toprule
        \textbf{Parametr}        & \textbf{Wartość} & \textbf{Napięcie sterujące [V]} \\
        \midrule
        Częstotliwość minimalna  & 282 Hz           & 5 V                             \\
        Częstotliwość maksymalna & 936 Hz           & 1,59 V                          \\
        \bottomrule
    \end{tabular}
\end{center}

\subsection{Zakres częstotliwości przy różnych nastawach}

Pomiary przy różnych nastawach potencjometru i napięcia sterującego:

\begin{center}
    \begin{tabular}{lccc}
        \toprule
        \textbf{Parametr}        & \textbf{Wartość [Hz]} & \textbf{Nastawa potencjometru} & \textbf{Napięcie sterujące [V]} \\
        \midrule
        Częstotliwość minimalna  & 218 Hz                & 10 k\ohm                       & 5 V                             \\
        Częstotliwość maksymalna & 983 Hz                & 3,1 k\ohm                      & 1,66 V                          \\
        \bottomrule
    \end{tabular}
\end{center}

\subsection{Charakterystyki częstotliwości i wypełnienia}

\noindent
\textbf{Nastawa potencjometru: $2 k\Omega$}

\begin{center}
    \begin{tabular}{ccc}
        \toprule
        \textbf{Napięcie sterujące [V]} & \textbf{Częstotliwość [Hz]} & \textbf{Wypełnienie [\%]} \\
        \midrule
        0,4                             & 252                         & 8,2                       \\
        1,2                             & 738                         & 17,3                      \\
        2,0                             & 916                         & 27,8                      \\
        2,98                            & 756                         & 42,9                      \\
        4,01                            & 534                         & 60,7                      \\
        4,71                            & 441                         & 70,6                      \\
        5,0                             & 361                         & 76,2                      \\
        \bottomrule
    \end{tabular}
\end{center}

\noindent
\textbf{Nastawa potencjometru: $7 k\Omega$}

\begin{center}
    \begin{tabular}{ccc}
        \toprule
        \textbf{Napięcie sterujące [V]} & \textbf{Częstotliwość [Hz]} & \textbf{Wypełnienie [\%]} \\
        \midrule
        0,05                            & 182                         & 11,1                      \\
        0,64                            & 476                         & 23,8                      \\
        1,21                            & 754                         & 29,6                      \\
        1,96                            & 831                         & 40,9                      \\
        2,73                            & 659                         & 53,3                      \\
        3,77                            & 435                         & 68,9                      \\
        4,43                            & 340                         & 77,3                      \\
        5,0                             & 240                         & 84,6                      \\
        \bottomrule
    \end{tabular}
\end{center}

\subsection{Wykresy charakterystyk}

\begin{figure}[H]
    \centering
    \begin{tikzpicture}
        \begin{axis}[ width=0.85\textwidth, height=7cm, xlabel={Napięcie sterujące $U$ [V]}, ylabel={Częstotliwość $f$ [Hz]}, grid=major, legend pos=north east, xmin=0, xmax=5.5, ymin=0, ymax=1000, mark size=2.5pt, ]

            % Dane dla r=2kΩ
            \addplot[color=xppblue, mark=*, thick] coordinates { (0.4, 252) (1.2, 738) (2.0, 916) (2.98, 756) (4.01, 534) (4.71, 441) (5.0, 361) }; \addlegendentry{$R = 2~\mathrm{k}\Omega$}

            % Dane dla r=7kΩ
            \addplot[color=red, mark=square*, thick] coordinates { (0.05, 182) (0.64, 476) (1.21, 754) (1.96, 831) (2.73, 659) (3.77, 435) (4.43, 340) (5.0, 240) }; \addlegendentry{$R = 7~\mathrm{k}\Omega$}

        \end{axis}
    \end{tikzpicture}
    \caption{Charakterystyka częstotliwości generowanego sygnału w funkcji napięcia sterującego}
    \label{fig:freq_char}
\end{figure}

\begin{figure}[H]
    \centering
    \begin{tikzpicture}
        \begin{axis}[ width=0.85\textwidth, height=7cm, xlabel={Napięcie sterujące $U$ [V]}, ylabel={Wypełnienie [\%]}, grid=major, legend pos=north west, xmin=0, xmax=5.5, ymin=0, ymax=90, mark size=2.5pt, ]

            % Dane dla r=2kΩ
            \addplot[color=xppblue, mark=*, thick] coordinates { (0.4, 8.2) (1.2, 17.3) (2.0, 27.8) (2.98, 42.9) (4.01, 60.7) (4.71, 70.6) (5.0, 76.2) }; \addlegendentry{$R = 2~\mathrm{k}\Omega$}

            % Dane dla r=7kΩ
            \addplot[color=red, mark=square*, thick] coordinates { (0.05, 11.1) (0.64, 23.8) (1.21, 29.6) (1.96, 40.9) (2.73, 53.3) (3.77, 68.9) (4.43, 77.3) (5.0, 84.6) }; \addlegendentry{$R = 7~\mathrm{k}\Omega$}

        \end{axis}
    \end{tikzpicture}
    \caption{Charakterystyka wypełnienia impulsów w funkcji napięcia sterującego}
    \label{fig:duty_char}
\end{figure}


\section{Badanie układu modulatora szerokości impulsów (PWM)}

\subsection{Charakterystyka przy maksymalnym $U_{DC}$}

Ustawienie maksymalizujące napięcie wyjściowe DC:

\vspace{1em}
\begin{center}
    \begin{tabular}{lcc}
        \toprule
        \textbf{Parametr}   & \textbf{Wartość} \\
        \midrule
        Napięcie sterujące  & 5 V              \\
        Maksymalne $U_{DC}$ & 3,7 V            \\
        \bottomrule
    \end{tabular}
\end{center}

\vspace{1em}

\subsection{Badanie charakterystyki układu}

\noindent
\vspace{1em}

\noindent
\textbf{a) Punkt równomiernego świecenia diod:}

Pomiar punktu równomiernego świecenia diod jest \textbf{subiektywny} ze względu na różną czułość oka ludzkiego na różne barwy światła (oko ludzkie jest bardziej czułe na światło zielone niż na czerwone). Teoretycznie, dla identycznych diod,
punkt równomiernego świecenia powinien występować przy wypełnieniu około \textbf{50\%}. W praktyce, jednak, może się różnić ze względu na różną czułość oka ludzkiego na różne barwy światła.

\vspace{1em}

\noindent
\textbf{b) Od jakich parametrów sygnału zależy jasność świecenia diod?}

Jasność świecenia diod zależy od wypełnienia sygnału PWM.

\vspace{0.5cm}

\subsection{Charakterystyka przy potencjometrze w położeniu środkowym}

\noindent
\textbf{Charakterystyka wypełnienia i składowej stałej:}

\begin{center}
    \begin{tabular}{cccc}
        \toprule
        \textbf{Napięcie sterujące [V]} & \textbf{Częstotliwość [Hz]} & \textbf{Wypełnienie [\%]} & \textbf{$U_{DC}$ [V]} \\
        \midrule
        4,92                            & 773                         & 88,65                     & 3,28                  \\
        4,37                            & 1053                        & 85,41                     & 3,16                  \\
        3,93                            & 1252                        & 81,62                     & 3,02                  \\
        3,40                            & 1667                        & 76,22                     & 2,82                  \\
        2,93                            & 2082                        & 70,27                     & 2,60                  \\
        2,52                            & 2483                        & 64,59                     & 2,39                  \\
        2,01                            & 3032                        & 56,49                     & 2,09                  \\
        1,52                            & 3448                        & 48,11                     & 1,78                  \\
        1,01                            & 2657                        & 41,62                     & 1,54                  \\
        0,50                            & 1781                        & 34,32                     & 1,27                  \\
        \bottomrule
    \end{tabular}
\end{center}

\subsection{Wykresy charakterystyk PWM}

\begin{figure}[H]
    \centering
    \begin{tikzpicture}
        \begin{axis}[ width=0.85\textwidth, height=7cm, xlabel={Napięcie sterujące $U$ [V]}, ylabel={Częstotliwość $f$ [Hz]}, grid=major, xmin=0, xmax=5.5, ymin=0, ymax=4000, mark size=2.5pt, ]

            \addplot[color=xppblue, mark=*, thick] coordinates { (0.50, 1781) (1.01, 2657) (1.52, 3448) (2.01, 3032) (2.52, 2483) (2.93, 2082) (3.40, 1667) (3.93, 1252) (4.37, 1053) (4.92, 773) };

        \end{axis}
    \end{tikzpicture}
    \caption{Charakterystyka częstotliwości modulatora PWM w funkcji napięcia sterującego (potencjometr w położeniu środkowym)}
    \label{fig:pwm_freq}
\end{figure}

\begin{figure}[H]
    \centering
    \begin{tikzpicture}
        \begin{axis}[ width=0.85\textwidth, height=7cm, xlabel={Napięcie sterujące $U$ [V]}, ylabel={Wypełnienie [\%]}, grid=major, xmin=0, xmax=5.5, ymin=0, ymax=100, mark size=2.5pt, ]

            \addplot[color=red, mark=square*, thick] coordinates { (0.50, 34.32) (1.01, 41.62) (1.52, 48.11) (2.01, 56.49) (2.52, 64.59) (2.93, 70.27) (3.40, 76.22) (3.93, 81.62) (4.37, 85.41) (4.92, 88.65) };

        \end{axis}
    \end{tikzpicture}
    \caption{Charakterystyka wypełnienia modulatora PWM w funkcji napięcia sterującego}
    \label{fig:pwm_duty}
\end{figure}

\begin{figure}[H]
    \centering
    \begin{tikzpicture}
        \begin{axis}[ width=0.85\textwidth, height=7cm, xlabel={Napięcie sterujące $U$ [V]}, ylabel={Napięcie DC $U_{DC}$ [V]}, grid=major, xmin=0, xmax=5.5, ymin=0, ymax=4, mark size=2.5pt, ]

            \addplot[color=green!60!black, mark=triangle*, thick] coordinates { (0.50, 1.27) (1.01, 1.54) (1.52, 1.78) (2.01, 2.09) (2.52, 2.39) (2.93, 2.60) (3.40, 2.82) (3.93, 3.02) (4.37, 3.16) (4.92, 3.28) };

        \end{axis}
    \end{tikzpicture}
    \caption{Charakterystyka składowej stałej napięcia wyjściowego $U_{DC}$ w funkcji napięcia sterującego}
    \label{fig:pwm_udc}
\end{figure}

\subsection{Zmienność częstotliwości w regulatorze PWM}

\noindent
\textbf{Czy w regulatorze PWM częstotliwość sygnału powinna się zmieniać? Jak dzieje się w badanym układzie?}

\vspace{0.5cm}

\noindent
W idealnym regulatorze PWM częstotliwość sygnału \textbf{nie powinna się zmieniać} -- powinna pozostać stała, a regulacja mocy następuje wyłącznie poprzez zmianę wypełnienia (duty cycle). Stała częstotliwość ułatwia filtrację i
przewidywalne zachowanie układu.

\vspace{0.5cm}

\noindent
W badanym układzie (opartym na czasówce 555) częstotliwość \textbf{zmienia się wraz z napięciem sterującym}, co jest wadą tego rozwiązania. Wynika to z budowy układu -- zmiana czasu ładowania/rozładowania kondensatora wpływa zarówno na
wypełnienie, jak i na okres sygnału. Jest to typowe ograniczenie prostych generatorów astabilnych.

\subsection{Zalety modulacji PWM}

\noindent
\textbf{Dlaczego stosuje się modulację PWM przy regulacji mocy układów DC zamiast regulacji ciągłej?}

\vspace{0.5cm}

\noindent
Modulacja PWM ma kilka kluczowych zalet nad regulacją ciągłą (np. rezystancyjną):

\vspace{0.3cm}

\noindent
\textbf{1. Wysoka sprawność energetyczna:} Tranzystor pracuje w dwóch stanach -- pełne przewodzenie (małe straty) lub zatkanie (brak prądu). W regulacji ciągłej element sterujący (np. tranzystor) pracuje w obszarze aktywnym, rozpraszając
dużo mocy jako ciepło.

\vspace{0.3cm}

\noindent
\textbf{2. Mniejsze straty mocy:} Energia nie jest marnowana na rezystancji -- jest po prostu "włączana" i "wyłączana" szybko, a odbiornik (np. silnik, LED) otrzymuje średnią wartość mocy.

\vspace{0.3cm}

\noindent
\textbf{3. Mniejsze nagrzewanie:} Dzięki minimalnym stratom mocy elementy sterujące nie wymagają dużych radiatorów chłodzących.

\vspace{0.3cm}

\noindent
\textbf{4. Lepsza kontrola:} PWM pozwala na precyzyjną i liniową regulację mocy w szerokim zakresie.

\section{Wnioski końcowe}

Podczas ćwiczenia zbadaliśmy układ NE555 pracujący jako generator VCO i modulator PWM. Z pomiarów wynika, że VCO działa liniowo tylko w zakresie napięć od 0 do około 1,25V -- powyżej tego częstotliwość zaczyna spadać zamiast rosnąć, więc nie
da się używać tego układu w pełnym zakresie napięć. W modulatorze PWM problem jest inny -- zmienia się nie tylko wypełnienie (co chcemy), ale też częstotliwość (czego nie chcemy), bo napięcie sterujące wpływa na czasy ładowania
kondensatora. Mimo tych wad układ dobrze reguluje moc (składowa stała $U_{DC}$ zmienia się liniowo z napięciem), co potwierdza że PWM jest znacznie wydajniejszy niż zwykła regulacja napięciem. W sumie NE555 jest prosty i tani, nadaje się do
podstawowych zastosowań, ale ma swoje ograniczenia przez nieliniowości i zależności od napięcia zasilania.

\end{document}
