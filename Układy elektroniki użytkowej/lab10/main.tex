\documentclass[11pt]{article}

\pdfobjcompresslevel=3    % compress PDF objects
\pdfcompresslevel=9       % compress streams more (0..9)
\pdfminorversion=4


\usepackage[utf8]{inputenc} % remove if using XeLaTeX or LuaLaTeX
\usepackage[T1]{fontenc}

% Layout & graphics
\usepackage[a4paper, total={7.5in, 9in}]{geometry}
\usepackage{graphicx}              % images
\usepackage[dvipsnames,table]{xcolor} % single xcolor invocation (keeps table option)

% Math & symbols
\usepackage{amsmath}
\usepackage{amssymb}
\usepackage{gensymb}

% Typography & layout helpers

\usepackage{polski}[babel]

\usepackage{microtype}
\usepackage{float}                 %  placement
\usepackage{caption}
\usepackage{subcaption}            % modern subfigure support (do NOT load subfig)
\usepackage{multirow}
\usepackage{titlesec}

\usepackage{lmodern}
\usepackage{microtype}

\usepackage{pgfplots}
\usepackage{booktabs}
\usepackage{siunitx}
\pgfplotsset{compat=1.18}

\definecolor{xppblue}{RGB}{37, 150, 190}

\begin{document}

\begin{table}
    \centering
    \scalebox{1.5}{
        \begin{tabular}{|c|p{3cm}|l|l|cl|}
            \hline
            \multicolumn{4}{|c|}{SPRAWOZDANIE Z LABORATORIUM}                         & \multicolumn{2}{c|}{\multirow{2}{*}{\begin{tabular}[c]{@{}c@{}}rok akademicki:\\ 2025/26\end{tabular}}}                                                          \\ \cline{1-4}
            \multicolumn{4}{|c|}{\textbf{Układy elektroniki użytkowej}}               & \multicolumn{2}{c|}{}                                                                                                                                            \\ \hline
            \multicolumn{4}{|c|}{\multirow{2}{*}{\textit{Relaksacyjny generator RC}}} & \multicolumn{2}{c|}{\multirow{2}{*}{czwartek 8:00}}                                                                                                              \\
            \multicolumn{4}{|c|}{}                                                    & \multicolumn{2}{c|}{}                                                                                                                                            \\ \hline
            \multicolumn{1}{|c|}{WARiE, AiR, sem 5}                                   & \multicolumn{3}{p{5cm}|}{\begin{tabular}[c]{@{}l@{}}1. Piotr Cybal\\ \underline{2. Piotr Bednarek}\end{tabular}} & \multicolumn{2}{l|}{\multirow{2}{*}{Punkty:}} \\ \cline{1-1}
            \multicolumn{1}{|c|}{08.12.2025}                                          & \multicolumn{3}{c|}{}                                                                                            & \multicolumn{2}{l|}{}                         \\ \hline
        \end{tabular}
    }
\end{table}

\vspace{1\baselineskip}

\section*{Sprzęt}
\begin{itemize}
    \item Numer stanowiska: 8
    \item Numer Elvisa: B
    \item Numer skrzynki z elementami: 3
\end{itemize}

\section{Cel ćwiczenia}

Celem ćwiczenia jest praktyczne zapoznanie się z zasadą działania relaksacyjnego
generatora RC sygnału prostokątnego, wykorzystującego w swojej strukturze wzmacniacz
operacyjny w konfiguracji przerzutnika Schmitta. W ramach laboratorium
przeprowadzona zostanie obserwacja oscyloskopowa przebiegów czasowych napięcia wyjściowego
oraz napięcia na kondensatorze, co pozwoli na analizę charakterystyki dynamicznej
układu.

\begin{figure}[H]
    \centering
    \includegraphics[width=0.5\textwidth]{DZIALA.png}
    \caption{Przykładowy przebieg czasowy napięcia na kondensatorze oraz napięcia wyjściowego generatora}
    \label{fig:przebiegi}
\end{figure}

Na powyższym rysunku przedstawiono charakterystyczny przebieg pracy generatora: napięcie
na kondensatorze narasta i opada wykładniczo, dążąc do napięć nasycenia wzmacniacza. W
momentach osiągnięcia progów przełączania przerzutnika Schmitta następuje gwałtowna zmiana
napięcia wyjściowego, co generuje sygnał prostokątny.

\section{Wzory}

\paragraph{Częstotliwość napięcia wyjściowego generatora relaksacyjnego RC}
\begin{equation}
    f_{gen} = \frac{1}{2RC \ln \frac{1+\beta}{1-\beta}}
\end{equation}
gdzie:
\begin{itemize}
    \item $\beta$ to stopień sprzężenia zwrotnego.
\end{itemize}

\paragraph{Stopień sprzężenia zwrotnego}
\begin{equation}
    \beta = \frac{U_{c,pom}}{U_{m,pom}}
\end{equation}
gdzie:
\begin{itemize}
    \item $U_{c,pom}$ to napięcie sterujące, w przybliżeniu równe napięciu kondensatora,
    \item $U_{m,pom}$ to międzyszczytowe napiecie wyjściowe generatora.
\end{itemize}

\section{Pomiary dla różnych konfiguracji układu RC}

Znaczące rozbieżności pomiędzy pomiarem częstotliwości generatora a obliczeniem
częstotliwości są spowodowane włączonym pomiarem dwóch kanałów oscyloskopu.
Zakłamuje to rzeczywistą wartość częstotliwości generatora, im wyższa jest częstotliwość
wyjściowa generatora tym Swiększa jest różnica między pomiarem a obliczeniem częstotliwości.


\subsection{1. Konfiguracja układu RC}

\begin{table}[H]
    \centering
    \caption{Wartości elementów RC dla pierwszej konfiguracji}
    \sisetup{
        output-decimal-marker = {,}
    }
    \begin{tabular}{cS[table-format=3.0]S[table-format=3.3]}
        \toprule
        Element & {Wartość nominalna}   & {Wartość zmierzona}     \\
        \midrule
        $R$     & \SI{20}{\kilo\ohm}    & \SI{19.866}{\kilo\ohm}  \\
        $C$     & \SI{100}{\nano\farad} & \SI{104,7}{\nano\farad} \\
        \bottomrule
    \end{tabular}
\end{table}

Obliczona stała czasowa: $\tau  = \SI{19,866}{\kilo\ohm} \cdot \SI{104,7}{\nano\farad} = \SI{2079,9702}{\micro\second}$

\begin{table}[H]
    \centering
    \caption{Wyniki pomiarów częstotliwości generatora w zależności od napięcia sterującego $U_c$}
    \sisetup{
        output-decimal-marker = {,}
    }
    \begin{tabular}{
            S[table-format=2.0]
            S[table-format=2.4]
            S[table-format=2.3]
            S[table-format=1.4]
            S[table-format=4.0]
            S[table-format=4.2]
        }
        \toprule
        \multicolumn{1}{c}{$U_c$ (V)} & \multicolumn{1}{c}{$U_{c,pom}$ (V)} & \multicolumn{1}{c}{$U_{m,pom}$ (V)} & \multicolumn{1}{c}{$\beta$} & \multicolumn{1}{c}{$f_{gen,pom}$ (Hz)} & \multicolumn{1}{c}{$f_{gen,obl}$ (Hz)} \\
        \midrule
        2                             & 1.999                               & 17.409                              & 0.1148                      & 1291                                   & 1083.81                                \\
        3                             & 3.036                               & 17.422                              & 0.1743                      & 829                                    & 709.99                                 \\
        4                             & 4.04                                & 17.43                               & 0.2318                      & 606                                    & 529.49                                 \\
        5                             & 4.993                               & 17.443                              & 0.2862                      & 473                                    & 424.49                                 \\
        6                             & 6.0006                              & 17.452                              & 0.3438                      & 349                                    & 348.74                                 \\
        7                             & 7.001                               & 17.469                              & 0.4008                      & 290                                    & 294.42                                 \\
        8                             & 7.996                               & 17.491                              & 0.4571                      & 246                                    & 253.19                                 \\
        9                             & 9.015                               & 17.501                              & 0.5151                      & 210                                    & 219.43                                 \\
        10                            & 10.013                              & 17.513                              & 0.5717                      & 181                                    & 192.27                                 \\
        11                            & 11.017                              & 17.522                              & 0.6288                      & 157                                    & 169.07                                 \\
        12                            & 12.011                              & 17.531                              & 0.6851                      & 137                                    & 149.04                                 \\
        13                            & 13.027                              & 17.541                              & 0.7427                      & 111                                    & 130.70                                 \\
        14                            & 14.001                              & 17.551                              & 0.7977                      & 96                                     & 114.43                                 \\
        15                            & 15.031                              & 17.565                              & 0.8557                      & 81                                     & 97.87                                  \\
        \bottomrule
    \end{tabular}
\end{table}


\subsection{2. Konfiguracja układu RC}

\begin{table}[H]
    \centering
    \caption{Wartości elementów RC dla drugiej konfiguracji}
    \sisetup{
        output-decimal-marker = {,}
    }
    \begin{tabular}{cS[table-format=3.0]S[table-format=3.3]}
        \toprule
        Element & {Wartość nominalna}  & {Wartość zmierzona}     \\
        \midrule
        $R$     & \SI{10}{\kilo\ohm}   & \SI{9,913}{\kilo\ohm}   \\
        $C$     & \SI{22}{\nano\farad} & \SI{21,56}{\nano\farad} \\
        \bottomrule
    \end{tabular}
\end{table}

Obliczona stała czasowa: $\tau  = \SI{9,913}{\kilo\ohm} \cdot \SI{21,56}{\nano\farad} = \SI{213,7243}{\micro\second}$


\begin{table}[H]
    \centering
    \caption{Wyniki pomiarów częstotliwości generatora w zależności od napięcia sterującego $U_c$ dla drugiej konfiguracji}
    \sisetup{
        output-decimal-marker = {,}
    }
    \begin{tabular}{
            S[table-format=2.0]
            S[table-format=2.3]
            S[table-format=2.3]
            S[table-format=1.4]
            S[table-format=5.0]
            S[table-format=5.2]
        }
        \toprule
        \multicolumn{1}{c}{$U_c$ (V)} & \multicolumn{1}{c}{$U_{c,pom}$ (V)} & \multicolumn{1}{c}{$U_{m,pom}$ (V)} & \multicolumn{1}{c}{$\beta$} & \multicolumn{1}{c}{$f_{gen,pom}$ (Hz)} & \multicolumn{1}{c}{$f_{gen,obl}$ (Hz)} \\
        \midrule
        2                             & 1.998                               & 17.244                              & 0.1159                      & 12180                                  & 9763.49                                \\
        3                             & 3.006                               & 17.269                              & 0.1741                      & 7999                                   & 6461.75                                \\
        4                             & 3.999                               & 17.291                              & 0.2313                      & 6051                                   & 4824.55                                \\
        5                             & 5.031                               & 17.317                              & 0.2905                      & 4753                                   & 3798.79                                \\
        6                             & 6.001                               & 17.336                              & 0.3462                      & 3951                                   & 3147.19                                \\
        7                             & 7.034                               & 17.356                              & 0.4053                      & 3311                                   & 2643.03                                \\
        8                             & 8.003                               & 17.381                              & 0.4604                      & 2959                                   & 2282.42                                \\
        9                             & 9.015                               & 17.405                              & 0.5180                      & 2575                                   & 1981.31                                \\
        10                            & 10.022                              & 17.426                              & 0.5751                      & 2164                                   & 1734.54                                \\
        11                            & 11.006                              & 17.447                              & 0.6308                      & 1912                                   & 1529.87                                \\
        12                            & 12.016                              & 17.472                              & 0.6877                      & 1682                                   & 1346.99                                \\
        13                            & 13.061                              & 17.496                              & 0.7465                      & 1466                                   & 1177.54                                \\
        14                            & 14.004                              & 17.515                              & 0.7995                      & 1289                                   & 1035.56                                \\
        15                            & 15.005                              & 17.546                              & 0.8552                      & 1104                                   & 891.18                                 \\
        \bottomrule
    \end{tabular}
\end{table}



\section{Charakterystyki częstotliwościowe}

\begin{figure}[H]
    \centering
    \begin{tikzpicture}
        \begin{axis}[
                width=0.9\textwidth,
                height=0.4\textheight,
                xlabel={Stopień sprzężenia zwrotnego $\beta$},
                ylabel={Częstotliwość $f_{gen}$ [Hz]},
                grid=major,
                legend pos=north east,
                title={Charakterystyka $f_{gen}(\beta)$ dla 1. konfiguracji RC}
            ]

            % Calculated points (Theory)
            \addplot[
                color=blue,
                mark=*,
                mark size=2pt,
                thick
            ]
            coordinates {
                    (0.1148, 1083.81)
                    (0.1743, 709.99)
                    (0.2318, 529.49)
                    (0.2862, 424.49)
                    (0.3438, 348.74)
                    (0.4008, 294.42)
                    (0.4571, 253.19)
                    (0.5151, 219.43)
                    (0.5717, 192.27)
                    (0.6288, 169.07)
                    (0.6851, 149.04)
                    (0.7427, 130.70)
                    (0.7977, 114.43)
                    (0.8557, 97.87)
                };
            \addlegendentry{Krzywa obliczona}

            % Measured Curve (Points + Line)
            \addplot[
                color=red,
                mark=square*,
                mark size=2pt,
                thick
            ]
            coordinates {
                    (0.1148, 1291)
                    (0.1743, 829)
                    (0.2318, 606)
                    (0.2862, 473)
                    (0.3438, 349)
                    (0.4008, 290)
                    (0.4571, 246)
                    (0.5151, 210)
                    (0.5717, 181)
                    (0.6288, 157)
                    (0.6851, 137)
                    (0.7427, 111)
                    (0.7977, 96)
                    (0.8557, 81)
                };
            \addlegendentry{Krzywa zmierzona}

        \end{axis}
    \end{tikzpicture}
    \caption{Wykres zależności częstotliwości generatora od stopnia sprzężenia zwrotnego dla 1. konfiguracji}
\end{figure}

\begin{figure}[H]
    \centering
    \begin{tikzpicture}
        \begin{axis}[
                width=0.9\textwidth,
                height=0.4\textheight,
                xlabel={Stopień sprzężenia zwrotnego $\beta$},
                ylabel={Częstotliwość $f_{gen}$ [Hz]},
                grid=major,
                legend pos=north east,
                title={Charakterystyka $f_{gen}(\beta)$ dla 2. konfiguracji RC}
            ]
            % Calculated points
            \addplot[
                color=blue,
                mark=*,
                mark size=2pt,
                thick
            ]
            coordinates {
                    (0.1159, 9763.49)
                    (0.1741, 6461.75)
                    (0.2313, 4824.55)
                    (0.2905, 3798.79)
                    (0.3462, 3147.19)
                    (0.4053, 2643.03)
                    (0.4604, 2282.42)
                    (0.5180, 1981.31)
                    (0.5751, 1734.54)
                    (0.6308, 1529.87)
                    (0.6877, 1346.99)
                    (0.7465, 1177.54)
                    (0.7995, 1035.56)
                    (0.8552, 891.18)
                };
            \addlegendentry{Wartości obliczone}

            \addplot[
                color=red,
                mark=square*,
                mark size=2pt,
                thick
            ]
            coordinates {
                    (0.1159, 12180)
                    (0.1741, 7999)
                    (0.2313, 6051)
                    (0.2905, 4753)
                    (0.3462, 3951)
                    (0.4053, 3311)
                    (0.4604, 2959)
                    (0.5180, 2575)
                    (0.5751, 2164)
                    (0.6308, 1912)
                    (0.6877, 1682)
                    (0.7465, 1466)
                    (0.7995, 1289)
                    (0.8552, 1104)
                };
            \addlegendentry{Krzywa zmierzona}
        \end{axis}
    \end{tikzpicture}
    \caption{Wykres zależności częstotliwości generatora od stopnia sprzężenia zwrotnego dla 2. konfiguracji}
\end{figure}

\section{Wpływ napięcia zasilania na pracę generatora}

W tabeli poniżej przedstawiono wyniki pomiarów napięcia wyjściowego generatora w zależności od napięcia zasilania.

\begin{table}[H]
    \centering
    \caption{Zależność amplitudy napięcia wyjściowego od napięcia zasilania}
    \sisetup{
        output-decimal-marker = {,}
    }
    \begin{tabular}{
            S[table-format=2.1]
            S[table-format=2.3]
        }
        \toprule
        {$U_{zas}$ (V)} & {$U_m$ (V)} \\
        \midrule
        10              & 17.555      \\
        9               & 15.614      \\
        8               & 13.661      \\
        7               & 11.703      \\
        6               & 9.761       \\
        5               & 7.818       \\
        4               & 5.877       \\
        3               & 3.945       \\
        2               & 2.015       \\
        \bottomrule
    \end{tabular}
\end{table}

Stwierdzono, że generator przestaje działać przy napięciu zasilania równym \SI{1,5}{\volt}.

\begin{figure}[H]
    \centering
    \includegraphics[width=0.5\textwidth]{NIE DZIALA.png}
    \caption{Oscylogram przedstawiający zanik oscylacji przy obniżonym napięciu zasilania}
    \label{fig:zanik}
\end{figure}

\section{Wnioski końcowe}

Przeprowadzone badania potwierdziły zgodność kształtu charakterystyki częstotliwościowej z
modelem teoretycznym, mimo występowania systematycznego przesunięcia wartości pomiarowych
względem obliczonych. Główną przyczyną tych rozbieżności jest dołączenie impedancji
wejściowej oscyloskopu, która modyfikuje rzeczywistą stałą czasową układu RC, co jest
szczególnie zauważalne przy wyższych częstotliwościach. Dodatkowo wykazano bezpośrednią
zależność amplitudy sygnału wyjściowego od napięcia zasilania oraz wyznaczono dolny próg
napięcia (ok. \SI{1,5}{\volt}), poniżej którego wzmacniacz wchodzi w nasycenie i generator
przestaje pracować.

\end{document}
