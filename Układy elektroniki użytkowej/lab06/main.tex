\documentclass[11pt]{article}

\pdfobjcompresslevel=3    % compress PDF objects
\pdfcompresslevel=9       % compress streams more (0..9)
\pdfminorversion=4


\usepackage[utf8]{inputenc} % remove if using XeLaTeX or LuaLaTeX
\usepackage[T1]{fontenc}

% Layout & graphics
\usepackage[a4paper, total={7.5in, 9in}]{geometry}
\usepackage{graphicx}              % images
\usepackage[dvipsnames,table]{xcolor} % single xcolor invocation (keeps table option)

% Math & symbols
\usepackage{amsmath}
\usepackage{amssymb}
\usepackage{gensymb}

% Typography & layout helpers

\usepackage{polski}[babel]

\usepackage{microtype}
\usepackage{float}                 %  placement
\usepackage{caption}
\usepackage{subcaption}            % modern subfigure support (do NOT load subfig)
\usepackage{multirow}
\usepackage{titlesec}

\usepackage{lmodern}
\usepackage{microtype}

\usepackage{pgfplots}
\usepackage{booktabs}
\usepackage{siunitx}
\pgfplotsset{compat=1.18}

\definecolor{xppblue}{RGB}{37, 150, 190}

\begin{document}

\begin{table}
    \centering
    \scalebox{1.5}{
        \begin{tabular}{|c|p{3cm}|l|l|cl|}
            \hline
            \multicolumn{4}{|c|}{SPRAWOZDANIE Z LABORATORIUM}                         & \multicolumn{2}{c|}{\multirow{2}{*}{\begin{tabular}[c]{@{}c@{}}rok akademicki:\\ 2025/26\end{tabular}}}                                                          \\ \cline{1-4}
            \multicolumn{4}{|c|}{\textbf{Układy elektroniki użytkowej}}               & \multicolumn{2}{c|}{}                                                                                                                                            \\ \hline
            \multicolumn{4}{|c|}{\multirow{2}{*}{\textit{Aktywne układy nieliniowe}}} & \multicolumn{2}{c|}{\multirow{2}{*}{czwartek 8:00}}                                                                                                              \\
            \multicolumn{4}{|c|}{}                                                    & \multicolumn{2}{c|}{}                                                                                                                                            \\ \hline
            \multicolumn{1}{|c|}{WARiE, AiR, sem 5}                                   & \multicolumn{3}{p{5cm}|}{\begin{tabular}[c]{@{}l@{}}1. Piotr Cybal\\ \underline{2. Piotr Bednarek}\end{tabular}} & \multicolumn{2}{l|}{\multirow{2}{*}{Punkty:}} \\ \cline{1-1}
            \multicolumn{1}{|c|}{12.11.2025}                                          & \multicolumn{3}{c|}{}                                                                                            & \multicolumn{2}{l|}{}                         \\ \hline
        \end{tabular}
    }
\end{table}

\vspace{1\baselineskip}

\section*{Sprzęt}
\begin{itemize}
    \item Numer stanowiska: 8
    \item Numer Elvisa: F
\end{itemize}

\section{Cel ćwiczenia}

Cel ćwiczenia Celem ćwiczenia było zapoznanie się z działaniem aktywnych układów
nieliniowych oraz analizą ich charakterystyk. W ramach zajęć zbadano:

\begin{itemize}
    \item Detektor szczytu pasywny – wpływ parametrów RC na dokładność detekcji i wzmocnienie w
          funkcji częstotliwości,
    \item Detektor szczytu aktywny – poprawę dokładności dzięki zastosowaniu wzmacniacza
          operacyjnego i wpływ elementów biernych na tętnienia
    \item Ogranicznik napięcia nieodwracający – charakterystykę przejściową i granice ograniczenia
          sygnału.
\end{itemize}

Pomiary umożliwiły zrozumienie zasad działania nieliniowych układów elektronicznych
stosowanych w kształtowaniu sygnałów analogowych.

\section{Detektor szczytu pasywny}
\subsection{Wykorzystane elementy}
\begin{itemize}
    \item Rezystor $R = 10 k\Omega$
    \item Kondensator $C = 0.1 \mu F$ (1J105K)
    \item Dioda PH4148
\end{itemize}

\subsection{Wykresy dla najwiekszej i najmniejszej częstotliwości}

\begin{figure}[H]
    \centering
    \begin{subfigure}{\textwidth}
        \centering
        \includegraphics[width=0.6\textwidth]{zdj/pasywny1.png}
        \caption{Częstotliwość 100 Hz}
        \label{fig:pasywny_100}
    \end{subfigure}
    \\[1cm]
    \begin{subfigure}{\textwidth}
        \centering
        \includegraphics[width=0.6\textwidth]{zdj/pasywny2.png}
        \caption{Częstotliwość 6400 Hz}
        \label{fig:pasywny_6400}
    \end{subfigure}
    \caption{Przebiegi napięcia wejściowego i wyjściowego detektora pasywnego}
    \label{fig:pasywny_przebiegi}
\end{figure}

\subsection{Tabela pomiarowa}

\begin{table}[H]
    \centering
    \caption{Pomiar maksymalnych napięć wejściowych i wyjściowych oraz wzmocnienia n dla różnych częstotliwości}
    \sisetup{
        output-decimal-marker = {,}
    }
    \begin{tabular}{
            S[table-format=4.0]  % f (Hz)
            S[table-format=1.2]  % U_we_max (V)
            S[table-format=1.2]  % U_wy_max (V)
            S[table-format=1.4]  % n
        }
        \toprule
        \multicolumn{1}{c}{f (Hz)} & \multicolumn{1}{c}{$U_{\mathrm{we,max}}$ (V)} & \multicolumn{1}{c}{$U_{\mathrm{wy,max}}$ (V)} & \multicolumn{1}{c}{n} \\
        \midrule
        100                        & 4.95                                          & 4.37                                          & 0.8828                \\
        200                        & 4.93                                          & 4.34                                          & 0.8803                \\
        400                        & 4.89                                          & 4.27                                          & 0.8732                \\
        800                        & 4.81                                          & 4.17                                          & 0.8669                \\
        1600                       & 4.75                                          & 4.09                                          & 0.8611                \\
        3200                       & 4.70                                          & 4.06                                          & 0.8638                \\
        6400                       & 4.71                                          & 4.02                                          & 0.8535                \\
        \bottomrule
    \end{tabular}
\end{table}

\begin{figure}[H]
    \centering
    \begin{tikzpicture}
        \begin{semilogxaxis}[
                width=0.8\textwidth,
                height=6cm,
                xlabel={Częstotliwość $f$ (Hz)},
                ylabel={Współczynnik $n$},
                legend pos={north east},
                grid=major,
                grid style={gray!30},
                xmin=50, xmax=10000,
                ymin=0.84, ymax=0.89,
                xtick={100,200,400,800,1600,3200,6400},
                xticklabels={100,200,400,800,1600,3200,6400},
            ]
            % Detektor pasywny
            \addplot[
                color=blue,
                mark=o,
                mark size=4pt,
                line width=1.5pt,
            ] coordinates {
                    (100, 0.8828)
                    (200, 0.8803)
                    (400, 0.8732)
                    (800, 0.8669)
                    (1600, 0.8611)
                    (3200, 0.8638)
                    (6400, 0.8535)
                };
            \addlegendentry{Detektor szczytu pasywny}
        \end{semilogxaxis}
    \end{tikzpicture}
    \caption{Wykes zależności współczynnika wzmocnienia $n$ od częstotliwości dla detektora pasywnego}
    \label{fig:pasywny}
\end{figure}

\vspace{1\baselineskip}

Na podstawie wykresu można stwierdzić, że współczynnik tętnień detektora szczytu
pasywnego maleje wraz ze wzrostem częstotliwości z około 0,88 do 0,85. Spadek o $~3-4\%$
wynika z niewystarczającego czasu ładowania kondensatora przy wyższych częstotliwościach –
kondensator nie zdąża się naładować do pełnej wartości szczytowej napięcia wejściowego.



\section{Detektor szczytu aktywny}
\subsection{Wykorzystane elementy}
\begin{itemize}
    \item Rezystor $R = 10 k\Omega$
    \item Kondensator $C = 0.1 \mu F$ (1J105K)
    \item Dioda PH4148
\end{itemize}

\subsection{Wykresy dla najwiekszej i najmniejszej częstotliwości}

\begin{figure}[H]
    \centering
    \begin{subfigure}{\textwidth}
        \centering
        \includegraphics[width=0.6\textwidth]{zdj/aktywny1.png}
        \caption{Częstotliwość 100 Hz}
        \label{fig:aktywny1}
    \end{subfigure}
    \\[1cm]
    \begin{subfigure}{\textwidth}
        \centering
        \includegraphics[width=0.6\textwidth]{zdj/aktywny2.png}
        \caption{Częstotliwość 6400 Hz}
        \label{fig:aktywny2}
    \end{subfigure}
    \caption{Przebiegi napięcia wejściowego i wyjściowego detektora aktywnego}
    \label{fig:aktywne+przebiegi}
\end{figure}


\begin{table}[H]
    \centering
    \caption{Pomiar maksymalnych napięć wejściowych i wyjściowych oraz wzmocnienia n dla różnych częstotliwości}
    \label{tab:measurements2}
    \sisetup{
        output-decimal-marker = {,}
    }
    \begin{tabular}{
            S[table-format=4.0]  % f (Hz)
            S[table-format=1.2]  % U_we_max (V)
            S[table-format=1.2]  % U_wy_max (V)
            S[table-format=1.4]  % n
        }
        \toprule
        \multicolumn{1}{c}{f (Hz)} & \multicolumn{1}{c}{$U_{\mathrm{we,max}}$ (V)} & \multicolumn{1}{c}{$U_{\mathrm{wy,max}}$ (V)} & \multicolumn{1}{c}{n} \\
        \midrule
        100                        & 4.98                                          & 4.98                                          & 1.0000                \\
        200                        & 4.97                                          & 4.97                                          & 1.0000                \\
        400                        & 4.98                                          & 4.97                                          & 0.9980                \\
        800                        & 4.97                                          & 4.97                                          & 1.0000                \\
        1600                       & 4.97                                          & 4.97                                          & 1.0000                \\
        3200                       & 4.95                                          & 4.90                                          & 0.9899                \\
        6400                       & 4.96                                          & 4.86                                          & 0.9798                \\
        \bottomrule
    \end{tabular}
\end{table}

\begin{figure}[H]
    \centering
    \begin{tikzpicture}
        \begin{semilogxaxis}[
                width=0.8\textwidth,
                height=6cm,
                xlabel={Częstotliwość $f$ (Hz)},
                ylabel={Współczynnik $n$},
                legend pos=north east,
                grid=major,
                grid style={gray!30},
                xmin=50, xmax=10000,
                ymin=0.96, ymax=1.02,
                xtick={100,200,400,800,1600,3200,6400},
                xticklabels={100,200,400,800,1600,3200,6400},
            ]
            % Detektor aktywny
            \addplot[
                color=red,
                mark=square,
                mark size=4pt,
                line width=1.5pt,
            ] coordinates {
                    (100, 1.0000)
                    (200, 1.0000)
                    (400, 0.9980)
                    (800, 1.0000)
                    (1600, 1.0000)
                    (3200, 0.9899)
                    (6400, 0.9798)
                };
            \addlegendentry{Detektor szczytu aktywny}
        \end{semilogxaxis}
    \end{tikzpicture}
    \caption{Wykes zależności współczynnika wzmocnienia $n$ od częstotliwości dla detektora aktywnego}
    \label{fig:aktywny}
\end{figure}

\subsection{Wpływ zmian wartości elementów na sygnał wynikowy}

\begin{figure}[H]
    \centering
    \centering
    \includegraphics[width=0.6\textwidth]{zdj/inny rezystor.png}
    \caption{Rezystor $R = 20 k\Omega$}
    \label{fig:inny_rezystor}
\end{figure}

\paragraph{Wpływ zmiany wartości rezystora na sygnał wynikowy}
W celu zbadania wpływu wartości rezystora na sygnał wyjściowy detektora szczytu aktywnego
przeprowadziliśmy pomiar dla rezystora o wartości $R = 20 k\Omega$.
\\
Zwiększenie oporu rezystora spowodowało, że rozładowanie kondensatora między szczytami
jest mniejsze dzięki czemu napięcie wyjściowe utrzymuje się bliżej wartości szczytowej i
tętnienia są zredukowane.

\paragraph{Wpływ zmiany wartości kondensatora na sygnał wynikowy}
Z kolei zwiększenie pojemności kondensatora wydłuża stałą czasową RC,
co również zmniejsza tętnienia napięcia wyjściowego i poprawia stabilność sygnału.
Zmniejszenie pojemności powoduje szybsze rozładowanie kondensatora między szczytami,
co zwiększa tętnienia i pogarsza jakość detekcji.

\paragraph{Dlaczego nie możemy użyć bardzo dużego rezystora?}
Zbyt duży opór rezystora może prowadzić do nadmiernego impedancji wyjściowej układu, co
utrudnia podłączenie obciążenia. Ponadto, bardzo duże wartości rezystorów są wrażliwe na


\section{Ogranicznik napiecia w konfiguracji nieodwracającej}
\subsection{Wykorzystane elementy}
\begin{itemize}
    \item Rezystor $R = 10 k\Omega$
\end{itemize}

\subsection{Tabela pomiarowa}

\begin{table}[H]
    \centering
    \caption{Pomiar napięć minimalnych i maksymalnych na wejściu i wyjściu ogranicznika napięcia}
    \label{tab:ogranicznik}
    \sisetup{
        round-mode = places,
        round-precision = 2,
        output-decimal-marker = {,}
    }
    \begin{tabular}{
            S[table-format=1.1]  % V_pp
            S[table-format=2.2]  % U_we_min (V)
            S[table-format=2.2]  % U_wy_min (V)
            S[table-format=2.2]  % U_we_max (V)
            S[table-format=2.2]  % U_wy_max (V)
        }
        \toprule
        {V$_{\mathrm{pp}}$} & {U$_{\mathrm{we,min}}$ (V)} & {U$_{\mathrm{wy,min}}$ (V)} & {U$_{\mathrm{we,max}}$ (V)} & {U$_{\mathrm{wy,max}}$ (V)} \\
        \midrule
        0.4                 & -0.10                       & -0.50                       & 0.09                        & 0.43                        \\
        0.8                 & -0.30                       & -1.50                       & 0.29                        & 1.44                        \\
        1.2                 & -0.51                       & -2.52                       & 0.49                        & 2.45                        \\
        1.6                 & -0.70                       & -3.49                       & 0.69                        & 3.42                        \\
        2.0                 & -0.90                       & -4.49                       & 0.89                        & 4.40                        \\
        2.4                 & -1.10                       & -5.49                       & 1.09                        & 5.41                        \\
        2.8                 & -1.30                       & -6.47                       & 1.29                        & 6.38                        \\
        3.2                 & -1.50                       & -7.47                       & 1.49                        & 7.38                        \\
        3.6                 & -1.71                       & -8.46                       & 1.69                        & 8.34                        \\
        4.0                 & -1.91                       & -9.21                       & 1.90                        & 9.03                        \\
        4.4                 & -2.10                       & -9.52                       & 2.09                        & 9.26                        \\
        4.8                 & -2.30                       & -9.75                       & 2.29                        & 9.48                        \\
        5.2                 & -2.51                       & -9.96                       & 2.49                        & 9.70                        \\
        5.6                 & -2.70                       & -10.17                      & 2.68                        & 9.90                        \\
        6.0                 & -2.90                       & -10.38                      & 2.88                        & 10.12                       \\
        6.4                 & -3.11                       & -10.55                      & 3.09                        & 10.33                       \\
        6.8                 & -3.31                       & -10.55                      & 3.28                        & 10.53                       \\
        7.2                 & -3.50                       & -10.55                      & 3.48                        & 10.56                       \\
        7.6                 & -3.70                       & -10.55                      & 3.68                        & 10.56                       \\
        \bottomrule
    \end{tabular}
\end{table}

\begin{figure}[H]
    \centering
    \begin{tikzpicture}
        \begin{axis}[
                width=0.9\textwidth,
                height=7cm,
                xlabel={Amplituda wejściowa $V_{\mathrm{pp}}$ (V)},
                ylabel={Napięcie (V)},
                % legend pos=upper left,
                % grid=minor,
                grid style={gray!30},
                xmin=0, xmax=8,
                ymin=-12, ymax=12,
            ]
            % U_we_min
            \addplot[
                color=blue,
                mark=o,
                mark size=3pt,
                line width=1.5pt,
            ] coordinates {
                    (0.4, -0.10)
                    (0.8, -0.30)
                    (1.2, -0.51)
                    (1.6, -0.70)
                    (2.0, -0.90)
                    (2.4, -1.10)
                    (2.8, -1.30)
                    (3.2, -1.50)
                    (3.6, -1.71)
                    (4.0, -1.91)
                    (4.4, -2.10)
                    (4.8, -2.30)
                    (5.2, -2.51)
                    (5.6, -2.70)
                    (6.0, -2.90)
                    (6.4, -3.11)
                    (6.8, -3.31)
                    (7.2, -3.50)
                    (7.6, -3.70)
                };
            \addlegendentry{U$_{\mathrm{we,min}}$}

            % U_wy_min
            \addplot[
                color=red,
                mark=square,
                mark size=3pt,
                line width=1.5pt,
            ] coordinates {
                    (0.4, -0.50)
                    (0.8, -1.50)
                    (1.2, -2.52)
                    (1.6, -3.49)
                    (2.0, -4.49)
                    (2.4, -5.49)
                    (2.8, -6.47)
                    (3.2, -7.47)
                    (3.6, -8.46)
                    (4.0, -9.21)
                    (4.4, -9.52)
                    (4.8, -9.75)
                    (5.2, -9.96)
                    (5.6, -10.17)
                    (6.0, -10.38)
                    (6.4, -10.55)
                    (6.8, -10.55)
                    (7.2, -10.55)
                    (7.6, -10.55)
                };
            \addlegendentry{U$_{\mathrm{wy,min}}$}

            % U_we_max
            \addplot[
                color=green!70!black,
                mark=triangle,
                mark size=3pt,
                line width=1.5pt,
            ] coordinates {
                    (0.4, 0.09)
                    (0.8, 0.29)
                    (1.2, 0.49)
                    (1.6, 0.69)
                    (2.0, 0.89)
                    (2.4, 1.09)
                    (2.8, 1.29)
                    (3.2, 1.49)
                    (3.6, 1.69)
                    (4.0, 1.90)
                    (4.4, 2.09)
                    (4.8, 2.29)
                    (5.2, 2.49)
                    (5.6, 2.68)
                    (6.0, 2.88)
                    (6.4, 3.09)
                    (6.8, 3.28)
                    (7.2, 3.48)
                    (7.6, 3.68)
                };
            \addlegendentry{U$_{\mathrm{we,max}}$}

            % U_wy_max
            \addplot[
                color=orange,
                mark=diamond,
                mark size=3pt,
                line width=1.5pt,
            ] coordinates {
                    (0.4, 0.43)
                    (0.8, 1.44)
                    (1.2, 2.45)
                    (1.6, 3.42)
                    (2.0, 4.40)
                    (2.4, 5.41)
                    (2.8, 6.38)
                    (3.2, 7.38)
                    (3.6, 8.34)
                    (4.0, 9.03)
                    (4.4, 9.26)
                    (4.8, 9.48)
                    (5.2, 9.70)
                    (5.6, 9.90)
                    (6.0, 10.12)
                    (6.4, 10.33)
                    (6.8, 10.53)
                    (7.2, 10.56)
                    (7.6, 10.56)
                };
            \addlegendentry{U$_{\mathrm{wy,max}}$}
        \end{axis}
    \end{tikzpicture}
    \caption{Zależność napięć minimalnych i maksymalnych na wejściu i wyjściu ogranicznika napięcia od amplitudy wejściowej}
    \label{fig:ogranicznik}
\end{figure}

\subsection{Ograniczenie sygnału wyjściowego}

\begin{figure}[H]
    \centering
    \includegraphics[width=0.6\textwidth]{zdj/detektor1.png}
    \caption{Przebiegi napięcia wejściowego i wyjściowego ogranicznika napięcia}
    \label{fig:detektor}
\end{figure}

Ograniczenie sygnału do około ±10,5 V wynika z napięcia zasilania wzmacniacza operacyjnego
(±12 V), który nie może wytworzyć napięcia większego niż jego zasilanie pomniejszone o
spadek nasycenia (~1,5 V).

\begin{figure}[H]
    \centering
    \includegraphics[width=0.6\textwidth]{zdj/22.png}
    \caption{Ograniczenia sygnałów dla tych samych wartości diod Zenera}
    \label{fig:zener_high}
\end{figure}



\section{Zastosowania omawianych układów}
\begin{itemize}
    \item \textbf{Detektor szczytu pasywny:}
          \begin{itemize}
              \item Proste układy pomiarowe sygnałów AC
              \item Zasilacze niskoprądowe (prostowniki szczytowe)
              \item Tanie rozwiązania tam, gdzie akceptowalne są niewielkie straty napięcia
          \end{itemize}

    \item \textbf{Detektor szczytu aktywny:}
          \begin{itemize}
              \item Precyzyjne pomiary wartości szczytowych sygnałów
              \item Układy próbkująco-pamiętające (sample-and-hold)
              \item Konwertery AC/DC w systemach akwizycji danych
              \item Demodulatory sygnałów AM
              \item Układy pomiarowe wymagające wysokiej dokładności
          \end{itemize}

    \item \textbf{Ogranicznik napięcia:}
          \begin{itemize}
              \item Ochrona wejść układów elektronicznych przed przepięciami
              \item Kształtowanie sygnałów (obcinanie amplitudy)
              \item Zabezpieczenia w analogowych torach przetwarzania sygnałów
              \item Generatory sygnałów o kontrolowanej amplitudzie
              \item Układy kompresji dynamiki sygnału
          \end{itemize}
\end{itemize}


\section{Wnioski}
\begin{itemize}
    \item Detektor szczytu pasywny: prosty i tani, lecz jego współczynnik przenoszenia
          maleje wraz z częstotliwością (ok. 0,88 → 0,85).
    \item Detektor szczytu aktywny: zapewnia współczynnik przenoszenia bliski jedności
          (~0,98–1,00) w szerokim paśmie. Zastosowanie wzmacniacza operacyjnego kompensuje
          spadek na diodzie i obniża impedancję wyjściową, co przekłada się na wyraźnie lepszą
          dokładność pomiaru.
    \item Wpływ elementów biernych: zwiększenie R lub C wydłuża stałą czasową RC i
          zmniejsza tętnienia wyjściowe, poprawiając stabilność detekcji. Należy jednak unikać
          nadmiernie dużych rezystancji ze względu na problemy z impedancją i podatność na
          zakłócenia.
    \item Ogranicznik napięcia: skutecznie ogranicza amplitudę do około ±10,5 V ze względu
          na zasilanie wzmacniacza (±12 V) oraz spadek nasycenia. W obszarze pracy liniowej
          układ wykazuje przewidywalne wzmocnienie, a przy przekroczeniu progu następuje
          oczekiwane obcięcie sygnału.

    \item Podsumowanie: przeprowadzone pomiary potwierdzają przewidywane zachowanie
          układów nieliniowych — detektory aktywne oferują lepszą dokładność i niższą impedancję
          wyjściową, natomiast pasywne rozwiązania są prostsze lecz mniej dokładne.
\end{itemize}




\end{document}
