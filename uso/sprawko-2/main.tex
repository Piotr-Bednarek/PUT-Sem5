\documentclass[12pt,a4paper]{article}

% Polskie znaki i język
\usepackage[utf8]{inputenc}
\usepackage[T1]{fontenc}
\usepackage[polish]{babel}
\usepackage{polski}

% Matematyka
\usepackage{amsmath, amssymb, amsfonts, mathtools}
\DeclareMathSymbol{.}{\mathord}{letters}{"3B}

% Grafika i wykresy
\usepackage{graphicx}
\usepackage{float}
\usepackage{wrapfig}
\usepackage{subcaption}

% Marginesy
\usepackage{geometry}
\geometry{ a4paper, left=25mm, right=25mm, top=25mm, bottom=25mm }

% Linki i odnośniki
\usepackage{hyperref}
\hypersetup{colorlinks=true, linkcolor=black, filecolor=magenta, urlcolor=blue, pdftitle={Sprawozdanie nr 2}}

% Kod źródłowy
\usepackage{listings}
\usepackage{xcolor}

% Definicja kolorów
\definecolor{codegreen}{rgb}{0,0.6,0}
\definecolor{codegray}{rgb}{0.5,0.5,0.5}
\definecolor{codepurple}{rgb}{0.58,0,0.82}
\definecolor{backcolour}{rgb}{0.95,0.95,0.92}

% Konfiguracja stylu dla Pythona
\lstdefinestyle{mystyle}{
  backgroundcolor=\color{backcolour},
  commentstyle=\color{codegreen},
  keywordstyle=\color{magenta},
  numberstyle=\tiny\color{codegray},
  stringstyle=\color{codepurple},
  basicstyle=\ttfamily\footnotesize,
  breakatwhitespace=false,
  breaklines=true,
  captionpos=b,
  keepspaces=true,
  numbers=left,
  numbersep=5pt,
  showspaces=false,
  showstringspaces=false,
  showtabs=false,
  tabsize=2
}
\lstset{style=mystyle}

% Nagłówki i stopki
\usepackage{fancyhdr}
\pagestyle{fancy}
\fancyhf{}
\setlength{\headheight}{15pt}
\rhead{Piotr Bednarek}
\lhead{Układy sterowania optymalnego}
\cfoot{\thepage}

% Tytuł i autor
\title{Sprawozdanie nr 2\\ \large Układy sterowania optymalnego}
\author{Piotr Bednarek}
\date{\today}

\begin{document}

\maketitle
\tableofcontents
\newpage

%% =============================================================================
%% ĆWICZENIE 10-11: Linearyzacja układów nieliniowych
%% =============================================================================

\section{Ćwiczenie 10-11: Linearyzacja układów nieliniowych}

\subsection{Linearyzacja względem punktu $x_0=(0,0), u_0=0$}

Postać zlinearyzowanego układu wahadła względem punktu $x_0=(0,0), u_0=0$:

\begin{equation}
  \begin{cases}
    \dot{x} = A x + B u \\
    A = \begin{bmatrix}
          0   & 1    \\
          -90 & -0.5
        \end{bmatrix}, \quad
    B = \begin{bmatrix}
          0 \\
          1
        \end{bmatrix}
  \end{cases}
\end{equation}

gdzie $A_{21} = -\frac{mgl}{J} = -90$ oraz $A_{22} = -\frac{d}{J} = -0.5$.

\begin{figure}[H]
  \centering
  \includegraphics[width=0.8\textwidth]{porownanie dla wymuszen.png}
  \caption{Porównanie odpowiedzi pierwszej zmiennej stanu (kąt $\theta$) układu nieliniowego
    i liniowego dla wymuszeń: $u = 0, 5, 20, 45\sqrt{2}$}
  \label{fig:linearyzacja_maleSrednie}
\end{figure}

\begin{figure}[H]
  \centering
  \includegraphics[width=0.8\textwidth]{duze-wymuszenie.png}
  \caption{Porównanie dla dużego wymuszenia ($u = 70$) – odpowiedź modelu liniowego rośnie w sposób nieograniczony, podczas gdy model nieliniowy pozostaje stabilny}
  \label{fig:linearyzacja_duze}
\end{figure}

\textbf{Komentarz do wyników:}

Na wykresach pokazano odpowiedź kąta $\theta$ dla układu nieliniowego (linie ciągłe) i zlinearyzowanego (linie przerywane) przy punkcie równowagi $x_0=(0,0)$, $u_0=0$. Wykresy dobrze pokazują, że linearyzacja działa tylko lokalnie:

\begin{itemize}
  \item \textbf{Małe wymuszenia} ($u = 0, 5, 20$): Model liniowy bardzo dobrze odzwierciedla model nieliniowy. Przebiegi są prawie identyczne, co potwierdza, że linearyzacja daje poprawne wyniki w otoczeniu punktu równowagi.

  \item \textbf{Średnie wymuszenie} ($u = 45\sqrt{2} \approx 63.6$): Pojawiają się małe różnice w amplitudach
        i wartościach ustalonych. Model liniowy nadal odwzorowuje charakter odpowiedzi, ale dokładność
        spada wraz z oddalaniem się od punktu linearyzacji.

  \item \textbf{Duże wymuszenie} ($u = 70$): Model liniowy nie pasuje do nieliniowego.
        Odpowiedź układu liniowego rośnie w sposób nieograniczony, podczas gdy nieliniowy zachowuje stabilne oscylacje. Świadczy to o tym, że aproksymacja liniowa jest poprawna jedynie w pobliżu punktu pracy.
\end{itemize}

\textbf{Interpretacja wymuszenia:}

Wymuszenie $u$ to moment siły działający na wahadło. Punkt równowagi $u_0 = 0$ oznacza,
że wahadło wisi pionowo w dół ($\theta = 0$). Większa wartość $u$ odpowiada większemu wychyleniu ustalonemu. \\

\textbf{Sterowalność układu:}

Układ jest sterowalny. Sprawdzono to wyznaczając macierz sterowalności:

\begin{equation}
  \mathcal{S} = [B \quad AB] = \begin{bmatrix}
    0 & 1    \\
    1 & -0.5
  \end{bmatrix}
\end{equation}

Wyznacznik: $\det(\mathcal{S}) = 0 \cdot (-0.5) - 1 \cdot 1 = -1 \neq 0$.

Ponieważ $\text{rank}(\mathcal{S}) = 2 = n$ (wymiar układu), układ jest w pełni sterowalny.

\subsection{Linearyzacja względem punktu $x_0=(\pi/4, 0), u_0=45\sqrt{2}$}

Postać zlinearyzowanego układu względem punktu $x_0=(\pi/4, 0), u_0=45\sqrt{2}$:

\begin{equation}
  \begin{cases}
    \dot{\tilde{x}} = A \tilde{x} + B \tilde{u} \\
    A = \begin{bmatrix}
          0           & 1    \\
          -45\sqrt{2} & -0.5
        \end{bmatrix} \approx \begin{bmatrix}
                                0      & 1    \\
                                -63.64 & -0.5
                              \end{bmatrix}, \quad
    B = \begin{bmatrix}
          0 \\
          1
        \end{bmatrix}
  \end{cases}
\end{equation}

gdzie $\tilde{x} = x - x_0$, $\tilde{u} = u - u_0$ są zmiennymi odchyłkowymi (przyrostowymi).

Macierz $A$ została obliczona jako Jakobian wokół punktu $(\pi/4, 0)$:
$$A_{21} = -\frac{mgl \cos(\theta_0)}{J} = -\frac{90 \cdot \cos(\pi/4)}{1} = -90 \cdot \frac{\sqrt{2}}{2} = -45\sqrt{2} \approx -63.64$$

\begin{figure}[H]
  \centering
  \includegraphics[width=0.85\textwidth]{./linearyzacja-pi4.png}
  \caption{Porównanie przebiegów pierwszej zmiennej stanu (kąt $\theta$) dla układu nieliniowego
    i zlinearyzowanego względem punktu równowagi $x_0=(\pi/4, 0)$ dla dwóch wartości
    stałego wymuszenia: $u = 45\sqrt{2}$ oraz $u = 45\sqrt{2}+10$}
  \label{fig:linearyzacja_pi4}
\end{figure}

\textbf{Odpowiedzi na pytania z zadania 5.2:}

\textbf{1) Jaka jest fizyczna interpretacja tego punktu równowagi?}

Punkt równowagi $x_0 = (\pi/4, 0)$ z $u_0 = 45\sqrt{2}$ oznacza, że wahadło jest odchylone
o $45^\circ$ od pionu (w dół) i spoczywa nieruchomo ($\dot{\theta} = 0$). Utrzymanie w tej pozycji
wymaga przykładania stałego momentu siły $u_0 = 45\sqrt{2} \approx 63.64$ [N$\cdot$m].

Moment ten równoważy grawitację dążącą do powrotu wahadła do pionu:
$$u_0 = mgl\sin(\pi/4) = 9 \cdot 10 \cdot 1 \cdot \frac{\sqrt{2}}{2} = 45\sqrt{2}$$

Jest to niestabilny punkt równowagi, co oznacza, że każde odchylenie wymaga korekty sterowania.

\textbf{2) Dlaczego czasami odpowiedź się nie zmienia?}

Kiedy $u = u_0 = 45\sqrt{2}$, układ znajduje się w stanie równowagi.
Wypadkowa sił działających na układ wynosi zero, zatem $\dot{\theta} = 0$ i $\ddot{\theta} = 0$.
Wahadło pozostaje w pozycji $\theta = \pi/4$ (zarówno w modelu nieliniowym, jak i liniowym).

Gdy $u \neq u_0$, równowaga zostaje zachwiana i wahadło zaczyna się poruszać.

\textbf{3) Czy odpowiedzi obu układów się pokrywają?}

\begin{itemize}
  \item Dla $u = u_0$ (równowaga): Odpowiedzi są identyczne -- stała wartość $\theta = \pi/4$.
        Wynika to bezpośrednio z definicji punktu równowagi.

  \item Dla $u = u_0 + 10$ (średnie odchylenie): Odpowiedzi są zbliżone, choć widoczne są niewielkie różnice.
        Model liniowy poprawnie przybliża dynamikę nieliniową w otoczeniu punktu pracy -- oba układy oscylują wokół nowej wartości ustalonej. Linearyzacja sprawdza się dla tego rzędu odchylenia ($\Delta u = 10$ względem $u_0 \approx 63.64$, czyli ok. 16\%).
\end{itemize}

\newpage
\textbf{Wnioski:}

Linearyzacja względem punktu $x_0 = (\pi/4, 0)$, $u_0 = 45\sqrt{2}$ jest skuteczna lokalnie. Dla wymuszeń bliskich $u_0$ oba modele dają zbieżne wyniki. Wykresy potwierdzają, że model liniowy jest przydatny do analizy i syntezy sterowania w ograniczonym obszarze wokół punktu pracy.

\subsection{Linearyzacja metodą SDC (zadanie 5.3)}

Metoda SDC (State-Dependent Coefficient) polega na zapisaniu układu nieliniowego w postaci pseudo-liniowej zależnej od stanu:
\begin{equation}
  \dot{x} = A(x)x + B(x)u
\end{equation}

\begin{lstlisting}[language=Python, caption=Funkcja implementujaca model SDC wahadla]
def wahadlo_sdc(x, t, u):
  kat = x[0]
  predkosc = x[1]

  if np.abs(kat) < 1e-10:
    sin = 1.0
  else:
    sin = np.sin(kat) / kat
  
  A = np.array([[0, 1],
                [-m*g*l/J * sin, -d/J]])
  B = np.array([[0], [1/J]])
  
  dxdt = A @ np.array([[kat], [predkosc]]) + B * u
  
  return dxdt.flatten()
\end{lstlisting}

\begin{figure}[H]
  \centering
  \includegraphics[width=0.8\textwidth]{sdc-test1.png}
  \caption{Porównanie układu nieliniowego i SDC dla $x_1(0) = \pi/4$, $x_2(0) = 0$, $u = 0$.
    Przebiegi są praktycznie identyczne -- maksymalny błąd wynosi $\approx 10^{-7}$ rad}
  \label{fig:sdc_test1}
\end{figure}

\begin{figure}[H]
  \centering
  \includegraphics[width=0.8\textwidth]{sdc-test2.png}
  \caption{Porównanie układu nieliniowego i SDC dla $x_1(0) = 0$, $x_2(0) = 0$, $u = 0$.
    Oba układy pozostają w punkcie równowagi -- błąd numeryczny wynosi 0}
  \label{fig:sdc_test2}
\end{figure}

\textbf{Komentarz do wyników SDC:}

Metoda SDC bardzo dokładnie odtwarza zachowanie oryginalnego układu nieliniowego. Uzyskane przebiegi są niemal identyczne, a obserwowane różnice wynikają wyłącznie z błędów numerycznych. Dla warunku początkowego $x_1(0) = \pi/4$ model SDC wiernie odwzorowuje tłumione oscylacje wahadła.

Kluczowym elementem implementacji jest obsługa punktu osobliwego $x_1 = 0$. W kodzie zastosowano warunek $|x_1| < \epsilon$, przyjmując w tym punkcie wartość graniczną $\lim_{x_1 \to 0} \frac{\sin(x_1)}{x_1} = 1$.

\textbf{Potencjalne ograniczenia metody SDC:}

\textbf{Ograniczenia teoretyczne:}

Nie każdy układ nieliniowy można zapisać w formie SDC -- równania stanu muszą posiadać odpowiednią strukturę (faktoryzowalność nieliniowości). Macierze $A(x)$ i $B(x)$ powinny być funkcjami gładkimi i ograniczonymi w przestrzeni stanów. Istotną cechą metody jest niejednoznaczność reprezentacji -- ten sam układ może być opisany różnymi parametryzacjami macierzy SDC, co wpływa na sterowalność modelu.

\textbf{Ograniczenia implementacyjne:}

Podstawowym problemem są punkty osobliwe, takie jak $x=0$ w przypadku wahadła, gdzie występuje dzielenie przez zero. Wymaga to jawnej obsługi wartości granicznych. Metoda jest również bardziej kosztowna obliczeniowo niż klasyczna linearyzacja, ponieważ macierze stanu są aktualizowane w każdym kroku całkowania. Należy też zadbać o stabilność numeryczną solvera przy szybkich zmianach stanu.

\textbf{Dlaczego może być błąd dla $x_1(0) = 0$?}

W punkcie $x_1 = 0$ wprost z równania wynikałoby dzielenie przez zero w członie $\frac{\sin(x_1)}{x_1}$. Choć granica tej funkcji wynosi 1, solver numeryczny bez odpowiedniego zabezpieczenia zwróciłby błąd (NaN). W implementacji rozwiązano to programowo, podstawiając wartość 1.0 dla bardzo małych kątów.

\newpage

%% =============================================================================
%% ĆWICZENIE 12: Regulator LQR dla układów nieliniowych
%% =============================================================================

\section{Ćwiczenie 12: Regulator LQR dla układów nieliniowych}

\subsection{Linearyzacja wahadła względem $x_0=(\pi, 0), u_0=0$}

Postać zlinearyzowanego układu wahadła względem punktu $x_0=(\pi, 0), u_0=0$:

\begin{equation}
  \begin{cases}
    \dot{x} = A x + B u \\
    A = \begin{bmatrix}
          0  & 1    \\
          90 & -0.5
        \end{bmatrix}, \quad
    B = \begin{bmatrix}
          0 \\
          1
        \end{bmatrix}
  \end{cases}
\end{equation}

\textbf{Dlaczego układ linearny jest niestabilny?}

Punkt $x_0 = (\pi, 0)$ odpowiada pozycji odwróconej (wahadło pionowo w górę), co jest fizycznie niestabilnym punktem równowagi. Analiza wartości własnych macierzy $A$:

$$\det(\lambda I - A) = \lambda(\lambda + 0.5) - 90 = \lambda^2 + 0.5\lambda - 90 = 0$$

prowadzi do pierwiastków:
$$\lambda_{1,2} = \frac{-0.5 \pm \sqrt{0.25 + 360}}{2} \approx \frac{-0.5 \pm 18.98}{2}$$

Stąd $\lambda_1 \approx 9.24$ (biegun niestabilny) oraz $\lambda_2 \approx -9.74$ (stabilny).
Dodatnia wartość własna oznacza, że odpowiedź układu liniowego na jakiekolwiek wymuszenie (np. skok) będzie rosła w sposób nieograniczony, co odzwierciedla niestabilność fizycznego obiektu.

\subsection{Regulator LQR dla wahadła}

Dla układu zlinearyzowanego wokół punktu $x_0 = (\pi, 0)$, $u_0 = 0$ zaprojektowano regulator
LQR z nieskończonym horyzontem czasowym. Macierze wag przyjęto jako:

$$Q = I_2 = \begin{bmatrix} 1 & 0 \\ 0 & 1 \end{bmatrix}, \quad R = 1$$

Wyznaczone wzmocnienia regulatora $K$:
\begin{equation}
  K = \begin{bmatrix}
    176.5 & 18.3
  \end{bmatrix}
\end{equation}

Prawo sterowania dla układu nieliniowego:
$$u = -K(x - x_0) + u_0$$

\begin{figure}[H]
  \centering
  \includegraphics[width=0.75\textwidth]{lqr-wahadlo-warunki.png}
  \caption{Przebiegi pierwszej zmiennej stanu (kąt $\theta$) układu zamkniętego z regulatorem LQR
    dla różnych warunków początkowych. Regulator stabilizuje wahadło w pozycji górnej ($\theta = \pi$)
    tylko dla małych odchyleń początkowych.}
  \label{fig:lqr_wahadlo_warunki}
\end{figure}

\textbf{Czy to rozwiązanie jest globalne?}

Nie, rozwiązanie jest lokalne. Regulator LQR, zaprojektowany na podstawie liniowej aproksymacji wokół punktu $(\pi, 0)$, stabilizuje nieliniowy układ tylko w pewnym otoczeniu tego punktu.

\textbf{Analiza wyników:}

\begin{itemize}
  \item Małe odchylenia ($\theta_0 = \pi - 0.1$, $\pi - 0.3$, $\pi - 0.5$):
        W symulacji regulator skutecznie stabilizuje wahadło w pozycji odwróconej. Układ asymptotycznie zmierza do punktu równowagi: $\theta \to \pi$, $\dot{\theta} \to 0$.

  \item Duże odchylenia ($\theta_0 = \pi/2$, $\theta_0 = 0$):
        Dla dużych odchyleń początkowych regulator nie jest w stanie ustabilizować obiektu. Wahadło opada do pozycji dolnej ($\theta \approx 0$) lub wykonuje ruch oscylacyjny. Wskazuje to, że dla takich wychyleń model liniowy, na podstawie którego wyznaczono wzmocnienia $K$, zbyt mocno odbiega od rzeczywistej dynamiki nieliniowej.
\end{itemize}

\textbf{Dlaczego rozwiązanie jest tylko lokalne:}

Linearyzacja modelu jest poprawna jedynie w niewielkim otoczeniu punktu pracy $(\pi, 0)$. Im większe odchylenie od pionu, tym większy błąd aproksymacji (wpływ nieliniowości $\sin \theta$). Z wykresów wynika, że obszar przyciągania regulatora wynosi około $|\theta - \pi| \approx 0.5$ rad ($30°$). Stabilizacja globalna wahadła odwróconego wymagałaby zastosowania nieliniowych algorytmów sterowania.

\textbf{Wnioski:}

Liniowy regulator kwadratowy (LQR) zastosowany do obiektu nieliniowego zapewnia stabilizację jedynie lokalnie. Jest to typowe ograniczenie metod projektowania opartych na linearyzacji w punkcie pracy.

\newpage
\subsection{Wykres z zadania 3.4}

Zbadano manipulator z regulatorem LQR w dwóch wariantach:
\begin{itemize}
  \item regulator z nieskończonym horyzontem czasowym (rozwiązanie algebraicznego równania Riccatiego - ARE),
  \item regulator ze skończonym horyzontem (równanie różniczkowe Riccatiego rozwiązywane wstecz na przedziale $[0, t_1]$).
\end{itemize}
Parametry: $J_1=J_2=0.5$, $k=100$, $c=1$, wagi $Q=I_4$, $R=1$, warunki początkowe $x(0) = [\pi,\ 0,\ \pi/2,\ 0]^T$.

\begin{figure}[H]
  \centering
  \includegraphics[width=0.75\textwidth]{lqr-manip-5s.png}
  \caption{Manipulator, $t_1=5$ s: porównanie pozycji ogniwa $x_1$ i silnika $x_3$ dla regulatora LQR z horyzontem nieskończonym i skończonym.}
  \label{fig:lqr_manip_5s}
\end{figure}

\begin{figure}[H]
  \centering
  \includegraphics[width=0.75\textwidth]{lqr-manip-15s.png}
  \caption{Manipulator, $t_1=1.5$ s: skrócony horyzont pogarsza jakość regulacji skończonej; LQR nieskończony nadal dąży do zera.}
  \label{fig:lqr_manip_15s}
\end{figure}

\begin{figure}[H]
  \centering
  \includegraphics[width=0.75\textwidth]{lqr-manip-S-sweep.png}
  \caption{Wpływ macierzy końcowej $S$ dla horyzontu $t_1=1.5$ s: zwiększenie kar $S$ poprawia zbieżność regulatora skończonego. Dla $S=100I$ przebiegi zbliżają się do rozwiązania z horyzontem nieskończonym.}
  \label{fig:lqr_manip_S}
\end{figure}

\textbf{Odpowiedzi na pytania:}

\begin{enumerate}
  \item \textbf{Czy oba regulatory LQR stabilizują układ?}
        Dla długiego horyzontu ($t_1=5$ s) tak. Oba regulatory skutecznie stabilizują układ, a ich przebiegi są niemal identyczne.

  \item \textbf{Czy zmienne zbiegają do zera po skróceniu horyzontu do $t_1=1.5$ s?}
        W tak krótkim czasie żaden z regulatorów nie sprowadza stanu idealnie do zera. Regulator nieskończony redukuje błąd, ale nie zdąży go wyeliminować. Regulator skończony przy domyślnym $S$ pozostawia znaczący uchyb końcowy, ponieważ horyzont sterowania kończy się zanim układ osiągnie równowagę.

  \item \textbf{Czy zmiana macierzy $S$ poprawia regulator skończony? Jaka wartość $S$ jest najlepsza?}
        Tak, zwiększenie wag w macierzy końcowej $S$ wymusza mniejszy błąd na końcu horyzontu sterowania. Najlepsze rezultaty uzyskano dla $S=100I$, gdzie przebiegi regulatora skończonego stają się bardzo zbliżone do regulatora z horyzontem nieskończonym. Mniejsze wartości ($S=0.1I$, $S=I$) dają gorszą zbieżność w krótkim horyzoncie czasu.
\end{enumerate}

\end{document}