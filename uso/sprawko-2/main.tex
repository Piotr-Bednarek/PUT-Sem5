\documentclass[12pt,a4paper]{article}

% Polskie znaki i język
\usepackage[utf8]{inputenc}
\usepackage[T1]{fontenc}
\usepackage[polish]{babel}
\usepackage{polski}

% Matematyka
\usepackage{amsmath, amssymb, amsfonts, mathtools}
\DeclareMathSymbol{.}{\mathord}{letters}{"3B}

% Grafika i wykresy
\usepackage{graphicx}
\usepackage{float}
\usepackage{wrapfig}
\usepackage{subcaption}

% Marginesy
\usepackage{geometry}
\geometry{ a4paper, left=25mm, right=25mm, top=25mm, bottom=25mm }

% Linki i odnośniki
\usepackage{hyperref}
\hypersetup{ colorlinks=true, linkcolor=black, filecolor=magenta,      
    urlcolor=blue, pdftitle={Sprawozdanie nr 2}, }

% Kod źródłowy
\usepackage{listings}
\usepackage{xcolor}

% Definicja kolorów
\definecolor{codegreen}{rgb}{0,0.6,0}
\definecolor{codegray}{rgb}{0.5,0.5,0.5}
\definecolor{codepurple}{rgb}{0.58,0,0.82}
\definecolor{backcolour}{rgb}{0.95,0.95,0.92}

% Konfiguracja stylu dla Pythona
\lstdefinestyle{mystyle}{ backgroundcolor=\color{backcolour},   
    commentstyle=\color{codegreen}, keywordstyle=\color{magenta},
    numberstyle=\tiny\color{codegray}, stringstyle=\color{codepurple},
    basicstyle=\ttfamily\footnotesize,
    breakatwhitespace=false,         
    breaklines=true,                 
    captionpos=b,                    
    keepspaces=true,                 
    numbers=left,                    
    numbersep=5pt,                  
    showspaces=false,                
    showstringspaces=false, showtabs=false,                  
    tabsize=2, literate={ą}{{\k{a}}}1 {ć}{{\'c}}1 {ę}{{\k{e}}}1 {ł}{{\l{}}}1 {ń}{{\'n}}1
    {ó}{{\'o}}1 {ś}{{\'s}}1 {ź}{{\'z}}1 {ż}{{\.z}}1 {Ą}{{\k{A}}}1 {Ć}{{\'C}}1
    {Ę}{{\k{E}}}1 {Ł}{{\L{}}}1 {Ń}{{\'N}}1 {Ó}{{\'O}}1 {Ś}{{\'S}}1 {Ź}{{\'Z}}1 {Ż}{{\.Z}}1
    }
\lstset{style=mystyle}

% Nagłówki i stopki
\usepackage{fancyhdr}
\pagestyle{fancy}
\fancyhf{}
\rhead{Piotr Bednarek}
\lhead{Układy sterowania optymalnego}
\cfoot{\thepage}

% Tytuł i autor
\title{Sprawozdanie nr 2\\ \large Układy sterowania optymalnego}
\author{Piotr Bednarek}
\date{\today}

\begin{document}

\maketitle
\tableofcontents
\newpage

%% =============================================================================
%% ĆWICZENIE 10-11: Linearyzacja układów nieliniowych
%% =============================================================================

\section{Ćwiczenie 10-11: Linearyzacja układów nieliniowych}

\subsection{Linearyzacja względem punktu $x_0=(0,0), u_0=0$}

% TODO: Wykres porównujący przebieg pierwszej zmiennej stanu dla układu 
% nieliniowego oraz zlinearyzowanego względem punktu x_0=(0,0), u_0=0

\begin{figure}[H]
    \centering
    % \includegraphics[width=0.75\textwidth]{img/linearyzacja_00.png}
    \caption{Porównanie przebiegów pierwszej zmiennej stanu dla układu nieliniowego
        oraz zlinearyzowanego względem punktu $x_0=(0,0), u_0=0$}
    \label{fig:linearyzacja_00}
\end{figure}

\textbf{Komentarz do wyników:}
% TODO: Skomentować uzyskane wyniki

\textbf{Sterowalność układu zlinearyzowanego:}
% TODO: Podać czy układ zlinearyzowany jest sterowalny oraz jaką metodę 
% badania sterowalności wykorzystano

\subsection{Linearyzacja względem punktu $x_0=(\pi/4, 0), u_0=45\sqrt{2}$}

Postać zlinearyzowanego układu względem punktu $x_0=(\pi/4, 0), u_0=45\sqrt{2}$:

% TODO: Podać postać zlinearyzowanego układu (macierze A, B)
\begin{equation}
    \begin{cases}
        \dot{x} = A x + B u \\
        A = \begin{bmatrix}
                \cdot & \cdot \\
                \cdot & \cdot
            \end{bmatrix}, \quad
        B = \begin{bmatrix}
                \cdot \\
                \cdot
            \end{bmatrix}
    \end{cases}
\end{equation}

\begin{figure}[H]
    \centering
    % \includegraphics[width=0.75\textwidth]{img/linearyzacja_pi4.png}
    \caption{Porównanie przebiegów pierwszej zmiennej stanu dla układu nieliniowego
        oraz zlinearyzowanego dla dwóch wartości stałego wymuszenia}
    \label{fig:linearyzacja_pi4}
\end{figure}

\textbf{Odpowiedzi na pytania z zadania 5.2:}
% TODO: Odpowiedzieć na pytania z zadania 5.2

\subsection{Linearyzacja metodą SDC (zadanie 5.3)}

\begin{lstlisting}[language=Python, caption=Kod do zadania 5.3 - linearyzacja metodą SDC]
# TODO: Zamieścić kod do zadania 5.3
\end{lstlisting}

\textbf{Komentarz do wyników dla linearyzacji metodą SDC:}
% TODO: Skomentować uzyskane wyniki dla linearyzacji metodą SDC

\textbf{Potencjalne ograniczenia metody SDC:}
% TODO: Jakie są potencjalne ograniczenia metody SDC?

\newpage

%% =============================================================================
%% ĆWICZENIE 12: Regulator LQR dla wahadła
%% =============================================================================

\section{Ćwiczenie 12: Regulator LQR dla wahadła}

\subsection{Linearyzacja wahadła względem $x_0=(\pi, 0), u_0=0$}

Postać zlinearyzowanego układu wahadła względem punktu $x_0=(\pi, 0), u_0=0$:

% TODO: Podać postać zlinearyzowanego układu wahadła (macierze A, B)
\begin{equation}
    \begin{cases}
        \dot{x} = A x + B u \\
        A = \begin{bmatrix}
                \cdot & \cdot \\
                \cdot & \cdot
            \end{bmatrix}, \quad
        B = \begin{bmatrix}
                \cdot \\
                \cdot
            \end{bmatrix}
    \end{cases}
\end{equation}

\textbf{Dlaczego zmienne stanu liniowej aproksymacji wahadła dążą do nieskończoności
    przy pobudzeniu skokiem jednostkowym?}
% TODO: Odpowiedź na pytanie z zadania 2.1

\subsection{Regulator LQR dla wahadła}

Wyznaczone wartości wzmocnień $K$ regulatora LQR:
% TODO: Podać wyznaczone wartości wzmocnień K regulatora LQR
\begin{equation}
    K = \begin{bmatrix}
        \cdot & \cdot
    \end{bmatrix}
\end{equation}

\begin{figure}[H]
    \centering
    % \includegraphics[width=0.75\textwidth]{img/lqr_wahadlo_warunki.png}
    \caption{Przebiegi pierwszej zmiennej stanu układu zamkniętego dla różnych
        warunków początkowych}
    \label{fig:lqr_wahadlo_warunki}
\end{figure}

\textbf{Czy wykorzystane rozwiązanie jest globalne?}
% TODO: Skomentować czy wykorzystane rozwiązanie jest globalne

\subsection{Wykres z zadania 3.4}

\begin{figure}[H]
    \centering
    % \includegraphics[width=0.75\textwidth]{img/zad_3_4.png}
    \caption{Wykres z zadania 3.4}
    \label{fig:zad_3_4}
\end{figure}

\textbf{Odpowiedzi na pytania z zadania 3.4:}
% TODO: Odpowiedzieć na postawione pytania z zadania 3.4

\textbf{Komentarz do wyników:}
% TODO: Skomentować uzyskane wyniki

\newpage

%% =============================================================================
%% ĆWICZENIE 13: Parametryzacja SDC dla wahadła
%% =============================================================================

\section{Ćwiczenie 13: Parametryzacja SDC dla wahadła}

\subsection{Propozycja parametryzacji SDC dla wahadła}

% TODO: Zaproponować parametryzację SDC dla wahadła
% Opisać zaproponowaną parametryzację

\textbf{Odpowiedzi na pytania z zadania 2.2:}
% TODO: Odpowiedzieć na pytania z zadania 2.2

\subsection{Wykresy z zadania 2.4}

\begin{figure}[H]
    \centering
    % \includegraphics[width=0.75\textwidth]{img/sdc_wahadlo_1.png}
    \caption{Wyniki z zadania 2.4 - część 1}
    \label{fig:sdc_wahadlo_1}
\end{figure}

% Jeśli potrzebne więcej wykresów:
% \begin{figure}[H]
%     \centering
%     \includegraphics[width=0.75\textwidth]{img/sdc_wahadlo_2.png}
%     \caption{Wyniki z zadania 2.4 - część 2}
%     \label{fig:sdc_wahadlo_2}
% \end{figure}

\textbf{Komentarz do wyników:}
% TODO: Skomentować uzyskane wyniki z zadania 2.4

\end{document}
