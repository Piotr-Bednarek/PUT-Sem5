\documentclass[12pt,a4paper]{article}

% Polskie znaki i język
\usepackage[utf8]{inputenc}
\usepackage[T1]{fontenc}
\usepackage[polish]{babel}
\usepackage{polski}

% Matematyka
\usepackage{amsmath, amssymb, amsfonts, mathtools}
\DeclareMathSymbol{.}{\mathord}{letters}{"3B}

% Grafika i wykresy
\usepackage{graphicx}
\usepackage{float}
\usepackage{wrapfig}
\usepackage{subcaption}

% Marginesy
\usepackage{geometry}
\geometry{ a4paper, left=25mm, right=25mm, top=25mm, bottom=25mm }

% Linki i odnośniki
\usepackage{hyperref}
\hypersetup{ colorlinks=true, linkcolor=black, filecolor=magenta,      
    urlcolor=blue, pdftitle={Sprawozdanie nr 2}, }

% Kod źródłowy
\usepackage{listings}
\usepackage{xcolor}

% Definicja kolorów
\definecolor{codegreen}{rgb}{0,0.6,0}
\definecolor{codegray}{rgb}{0.5,0.5,0.5}
\definecolor{codepurple}{rgb}{0.58,0,0.82}
\definecolor{backcolour}{rgb}{0.95,0.95,0.92}

% Konfiguracja stylu dla Pythona
\lstdefinestyle{mystyle}{ backgroundcolor=\color{backcolour},   
    commentstyle=\color{codegreen}, keywordstyle=\color{magenta},
    numberstyle=\tiny\color{codegray}, stringstyle=\color{codepurple},
    basicstyle=\ttfamily\footnotesize,
    breakatwhitespace=false,         
    breaklines=true,                 
    captionpos=b,                    
    keepspaces=true,                 
    numbers=left,                    
    numbersep=5pt,                  
    showspaces=false,                
    showstringspaces=false, showtabs=false,                  
    tabsize=2, literate={ą}{{\k{a}}}1 {ć}{{\'c}}1 {ę}{{\k{e}}}1 {ł}{{\l{}}}1 {ń}{{\'n}}1
    {ó}{{\'o}}1 {ś}{{\'s}}1 {ź}{{\'z}}1 {ż}{{\.z}}1 {Ą}{{\k{A}}}1 {Ć}{{\'C}}1
    {Ę}{{\k{E}}}1 {Ł}{{\L{}}}1 {Ń}{{\'N}}1 {Ó}{{\'O}}1 {Ś}{{\'S}}1 {Ź}{{\'Z}}1 {Ż}{{\.Z}}1
    }
\lstset{style=mystyle}

% Nagłówki i stopki
\usepackage{fancyhdr}
\pagestyle{fancy}
\fancyhf{}
\rhead{Piotr Bednarek}
\lhead{Układy sterowania optymalnego}
\cfoot{\thepage}

% Tytuł i autor
\title{Sprawozdanie nr 2\\ \large Układy sterowania optymalnego}
\author{Piotr Bednarek}
\date{\today}

\begin{document}

\maketitle
\tableofcontents
\newpage

%% =============================================================================
%% ĆWICZENIE 10-11: Linearyzacja układów nieliniowych
%% =============================================================================

\section{Ćwiczenie 10-11: Linearyzacja układów nieliniowych}

\subsection{Linearyzacja względem punktu $x_0=(0,0), u_0=0$}

Postać zlinearyzowanego układu wahadła względem punktu $x_0=(0,0), u_0=0$:

\begin{equation}
  \begin{cases}
    \dot{x} = A x + B u \\
    A = \begin{bmatrix}
          0   & 1    \\
          -90 & -0.5
        \end{bmatrix}, \quad
    B = \begin{bmatrix}
          0 \\
          1
        \end{bmatrix}
  \end{cases}
\end{equation}

gdzie $A_{21} = -\frac{mgl}{J} = -90$ oraz $A_{22} = -\frac{d}{J} = -0.5$.

\begin{figure}[H]
  \centering
  \includegraphics[width=0.8\textwidth]{porownanie dla wymuszen.png}
  \caption{Porównanie odpowiedzi pierwszej zmiennej stanu (kąt $\theta$) układu nieliniowego
    i liniowego dla wymuszeń: $u = 0, 5, 20, 45\sqrt{2}$}
  \label{fig:linearyzacja_maleSrednie}
\end{figure}

\begin{figure}[H]
  \centering
  \includegraphics[width=0.8\textwidth]{duze-wymuszenie.png}
  \caption{Porównanie dla dużego wymuszenia ($u = 70$) - model liniowy eksploduje do nieskończoności,
    podczas gdy model nieliniowy pozostaje stabilny}
  \label{fig:linearyzacja_duze}
\end{figure}

\textbf{Komentarz do wyników:}

Na wykresach przedstawiono odpowiedź pierwszej zmiennej stanu (kąt $\theta$) dla układu nieliniowego
(linie ciągłe) oraz zlinearyzowanego (linie przerywane) względem punktu równowagi $x_0=(0,0)$, $u_0=0$.
Wykresy jednoznacznie pokazują lokalny charakter linearyzacji:

\begin{itemize}
  \item \textbf{Małe wymuszenia} ($u = 0, 5, 20$): Model liniowy wiernie odzwierciedla zachowanie
        nieliniowego obiektu. Przebiegi obu modeli są niemal identyczne, co potwierdza prawidłowość
        linearyzacji w okolicy punktu równowagi.

  \item \textbf{Średnie wymuszenie} ($u = 45\sqrt{2} \approx 63.6$): Pojawiają się niewielkie różnice w amplitudach
        i wartościach ustalonych. Model liniowy nadal zachowuje ogólny charakter odpowiedzi, ale dokładność
        się pogarsza wraz z oddalaniem od punktu linearyzacji.

  \item \textbf{Duże wymuszenie} ($u = 70$): Model zlinearyzowany nie odzwierciedla zachowania
        nieliniowego obiektu. Liniowy eksploduje do dużych wartości, podczas gdy nieliniowy
        pozostaje stabilny z oscylacjami tłumionymi. To potwierdza, że linearyzacja jest wiarygodna
        tylko lokalnie - dla małych odchyleń od punktu równowagi.
\end{itemize}

\textbf{Interpretacja wartości wymuszenia:}

Wymuszenie $u$ reprezentuje moment siły działający na wahadło. Punkt równowagi $u_0 = 0$ odpowiada
stanowi, w którym wahadło wisi pionowo w dół ($\theta = 0$). Im większe wymuszenie, tym większe
wychylenie wahadła od pozycji równowagi. \\

\textbf{Sterowalność układu zlinearyzowanego:}

Układ jest sterowalny. Do badania sterowalności wykorzystano macierz sterowalności:

\begin{equation}
  \mathcal{S} = [B \quad AB] = \begin{bmatrix}
    0 & 1    \\
    1 & -0.5
  \end{bmatrix}
\end{equation}

Wyznacznik macierzy sterowalności: $\det(\mathcal{S}) = 0 \cdot (-0.5) - 1 \cdot 1 = -1 \neq 0$

Ponieważ $\text{rank}(\mathcal{S}) = 2 = n$ (wymiar układu), układ jest w pełni sterowalny.

\subsection{Linearyzacja względem punktu $x_0=(\pi/4, 0), u_0=45\sqrt{2}$}

Postać zlinearyzowanego układu względem punktu $x_0=(\pi/4, 0), u_0=45\sqrt{2}$:

\begin{equation}
  \begin{cases}
    \dot{\tilde{x}} = A \tilde{x} + B \tilde{u} \\
    A = \begin{bmatrix}
          0           & 1    \\
          -45\sqrt{2} & -0.5
        \end{bmatrix} \approx \begin{bmatrix}
                                0      & 1    \\
                                -63.64 & -0.5
                              \end{bmatrix}, \quad
    B = \begin{bmatrix}
          0 \\
          1
        \end{bmatrix}
  \end{cases}
\end{equation}

gdzie $\tilde{x} = x - x_0$, $\tilde{u} = u - u_0$ są zmiennymi odchyłkowymi (przyrostowymi).

Macierz $A$ została obliczona jako Jakobian wokół punktu $(\pi/4, 0)$:
$$A_{21} = -\frac{mgl \cos(\theta_0)}{J} = -\frac{90 \cdot \cos(\pi/4)}{1} = -90 \cdot \frac{\sqrt{2}}{2} = -45\sqrt{2} \approx -63.64$$

\begin{figure}[H]
  \centering
  \includegraphics[width=0.85\textwidth]{./linearyzacja-pi4.png}
  \caption{Porównanie przebiegów pierwszej zmiennej stanu (kąt $\theta$) dla układu nieliniowego
    i zlinearyzowanego względem punktu równowagi $x_0=(\pi/4, 0)$ dla dwóch wartości
    stałego wymuszenia: $u = 45\sqrt{2}$ oraz $u = 45\sqrt{2}+10$}
  \label{fig:linearyzacja_pi4}
\end{figure}

\textbf{Odpowiedzi na pytania z zadania 5.2:}

\textbf{1) Jaka jest fizyczna interpretacja tego punktu równowagi?}

Punkt równowagi $x_0 = (\pi/4, 0)$ z $u_0 = 45\sqrt{2}$ odpowiada stanowi, w którym wahadło jest odchylone
o $45°$ od pionu (w dół) i pozostaje w spoczynku ($\dot{\theta} = 0$). Aby utrzymać wahadło w tej pozycji,
konieczne jest przyłożenie stałego momentu siły $u_0 = 45\sqrt{2} \approx 63.64$ [N$\cdot$m].

Moment ten równoważy składową grawitacji dążącą do powrotu wahadła do pozycji pionowej w dół:
$$u_0 = mgl\sin(\pi/4) = 9 \cdot 10 \cdot 1 \cdot \frac{\sqrt{2}}{2} = 45\sqrt{2}$$

Jest to niestabilny punkt równowagi -- każde odchylenie wymaga skorygowania momentu $u$.

\textbf{2) Dlaczego w niektórych przypadkach odpowiedź układu nie zmienia się w czasie?}

Dla wymuszenia $u = u_0 = 45\sqrt{2}$ układ znajduje się dokładnie w punkcie równowagi.
Żadna siła wypadkowa nie działa na układ, zatem $\dot{\theta} = 0$ i $\ddot{\theta} = 0$.
Wahadło pozostaje w pozycji $\theta = \pi/4$ przez cały czas symulacji (zarówno w modelu
nieliniowym, jak i liniowym).

Dla pozostałych wymuszeń ($u \neq u_0$) pojawia się niezerowa siła wypadkowa, co powoduje
ruch wahadła.

\textbf{3) Czy odpowiedzi obu układów się pokrywają? Dlaczego?}

\begin{itemize}
  \item Dla $u = u_0$ (równowaga): Odpowiedzi są identyczne -- obie wynoszą
        stale $\theta = \pi/4$. To oczywiste, gdyż oba układy znajdują się w punkcie równowagi.
        Żadna siła wypadkowa nie działa, więc kąt pozostaje stały.

  \item Dla $u = u_0 + 10$ (średnie odchylenie): Odpowiedzi pokrywają się dobrze,
        ale widoczne są niewielkie różnice. Model liniowy wiernie przybliża zachowanie nieliniowe
        w okolicy punktu równowagi -- obie odpowiedzi oscylują wokół wartości ustalonej około
        $0.96$ rad. Linearyzacja zachowuje się poprawnie dla tego stosunkowo niewielkiego odchylenia
        od punktu równowagi ($\Delta u = 10$ względem $u_0 \approx 63.64$, czyli około 16\%).
\end{itemize}

\textbf{Wnioski:}

Linearyzacja względem punktu $x_0 = (\pi/4, 0)$, $u_0 = 45\sqrt{2}$ jest poprawna lokalnie -- w okolicy
punktu równowagi. Dla wymuszeń bliskich $u_0$ oba modele dają zbieżne wyniki. Wykres pokazuje, że
linearyzacja działa dobrze nawet dla średnich odchyleń (16\% zmiany wymuszenia), co potwierdza jej
przydatność do analizy i projektowania układów sterowania w ograniczonym obszarze pracy.

\subsection{Linearyzacja metodą SDC (zadanie 5.3)}

Metoda SDC (State-Dependent Coefficient) polega na zapisaniu układu nieliniowego w postaci
liniowej zależnej od stanu:
\begin{equation}
  \dot{x} = A(x)x + B(x)u
\end{equation}

\begin{lstlisting}[language=Python, caption=Funkcja implementująca model SDC wahadła]
def wahadlo_sdc(x, t, u):
    kat = x[0]
    predkosc = x[1]
    
    if np.abs(kat) < 1e-10:
        sin = 1.0
    else:
        sin = np.sin(kat) / kat
    
    A = np.array([[0, 1],
                  [-m*g*l/J * sin, -d/J]])
    B = np.array([[0], [1/J]])
    
    dxdt = A @ np.array([[kat], [predkosc]]) + B * u
    
    return dxdt.flatten()
\end{lstlisting}

\begin{figure}[H]
  \centering
  \includegraphics[width=0.8\textwidth]{sdc-test1.png}
  \caption{Porównanie układu nieliniowego i SDC dla $x_1(0) = \pi/4$, $x_2(0) = 0$, $u = 0$.
    Przebiegi są praktycznie identyczne -- maksymalny błąd wynosi $\approx 10^{-7}$ rad}
  \label{fig:sdc_test1}
\end{figure}

\begin{figure}[H]
  \centering
  \includegraphics[width=0.8\textwidth]{sdc-test2.png}
  \caption{Porównanie układu nieliniowego i SDC dla $x_1(0) = 0$, $x_2(0) = 0$, $u = 0$.
    Oba układy pozostają w punkcie równowagi -- błąd numeryczny wynosi 0}
  \label{fig:sdc_test2}
\end{figure}

\textbf{Komentarz do wyników dla linearyzacji metodą SDC:}

Wyniki jednoznacznie pokazują, że metoda SDC dokładnie reprodukuje zachowanie oryginalnego
układu nieliniowego. Przebiegi są praktycznie identyczne -- różnice są na poziomie błędów
numerycznych

Dla warunku początkowego $x_1(0) = \pi/4$ wahadło wykonuje tłumione oscylacje wokół pozycji
równowagi $\theta = 0$, a układ SDC wiernie odtwarza tę dynamikę. Dla $x_1(0) = 0$ oba układy
pozostają w punkcie równowagi.

Istotna jest prawidłowa obsługa punktu osobliwego $x_1 = 0$. Implementacja wykorzystuje
warunek $|x_1| < \epsilon$, który zwraca wartość graniczną $\lim_{x_1 \to 0} \frac{\sin(x_1)}{x_1} = 1$.
\newpage
\textbf{Potencjalne ograniczenia metody SDC:}

\textbf{Ograniczenia teoretyczne:}
\begin{itemize}
  \item Nie każdy układ nieliniowy można zapisać w formie SDC -- wymaga to odpowiedniej
        struktury równań
  \item Macierze $A(x)$ i $B(x)$ muszą być funkcjami gładkimi (ciągłe i różniczkowalne)
  \item Współczynniki macierzy muszą pozostać ograniczone dla wszystkich możliwych stanów $x$
  \item Brak jednoznaczności -- ten sam układ nieliniowy może mieć wiele różnych reprezentacji SDC
\end{itemize}

\textbf{Ograniczenia implementacyjne:}
\begin{itemize}
  \item Potencjalne problemy numeryczne w punktach osobliwych (np. dzielenie przez zero)
  \item Konieczność obsługi wartości granicznych funkcji (np. $\lim_{x \to 0} \frac{\sin(x)}{x}$)
  \item Większe koszty obliczeniowe -- macierze $A(x)$ i $B(x)$ muszą być obliczane w każdym
        kroku całkowania (w przeciwieństwie do stałych macierzy w linearyzacji Jakobianowskiej)
  \item Wymaga starannej implementacji dla zapewnienia stabilności i dokładności numerycznej
\end{itemize}

\textbf{Przyczyna błędu/ostrzeżenia dla $x_1(0) = 0$:}

W punkcie $x_1 = 0$ pojawia się dzielenie przez zero w wyrażeniu $\frac{\sin(x_1)}{x_1}$.
Matematycznie $\lim_{x_1 \to 0} \frac{\sin(x_1)}{x_1} = 1$, ale bez specjalnej obsługi
solver numeryczny wygenerowałby ostrzeżenie dzielenia przez zero. Implementacja rozwiązuje ten problem przez sprawdzenie
warunku $|x_1| < 10^{-10}$ i zwrócenie wartości $1.0$ dla małych $x_1$.

\newpage

%% =============================================================================
%% ĆWICZENIE 12: Regulator LQR dla układów nieliniowych
%% =============================================================================

\section{Ćwiczenie 12: Regulator LQR dla układów nieliniowych}

\subsection{Linearyzacja wahadła względem $x_0=(\pi, 0), u_0=0$}

Postać zlinearyzowanego układu wahadła względem punktu $x_0=(\pi, 0), u_0=0$:

\begin{equation}
  \begin{cases}
    \dot{x} = A x + B u \\
    A = \begin{bmatrix}
          0  & 1    \\
          90 & -0.5
        \end{bmatrix}, \quad
    B = \begin{bmatrix}
          0 \\
          1
        \end{bmatrix}
  \end{cases}
\end{equation}

\textbf{Dlaczego zmienne stanu liniowej aproksymacji wahadła dążą do nieskończoności
  przy pobudzeniu skokiem jednostkowym?}

Układ zlinearyzowany względem punktu $x_0 = (\pi, 0)$ jest niestabilny. Punkt ten
odpowiada pozycji wahadła skierowanego pionowo w górę (pozycja odwrócona), co stanowi
niestabilny punkt równowagi. Wyznaczając wartości własne macierzy $A$:

$$\det(\lambda I - A) = \lambda(\lambda + 0.5) - 90 = \lambda^2 + 0.5\lambda - 90 = 0$$

otrzymujemy:
$$\lambda_{1,2} = \frac{-0.5 \pm \sqrt{0.25 + 360}}{2} = \frac{-0.5 \pm 18.98}{2}$$

co daje: $\lambda_1 \approx 9.24$ (niestabilny biegun) oraz $\lambda_2 \approx -9.74$ (stabilny).

Obecność wartości własnej o dodatniej części rzeczywistej ($\lambda_1 > 0$) powoduje,
że odpowiedź układu rośnie wykładniczo w czasie, dążąc do nieskończoności. Fizycznie
oznacza to, że wahadło w pozycji odwróconej jest niestabilne -- każde najmniejsze
odchylenie (w tym wymuszenie skokowe) powoduje jego upadek. Model liniowy odzwierciedla
tę niestabilność.

\subsection{Regulator LQR dla wahadła}

Dla układu zlinearyzowanego wokół punktu $x_0 = (\pi, 0)$, $u_0 = 0$ zaprojektowano regulator
LQR z nieskończonym horyzontem czasowym. Przyjęto macierze wag:

$$Q = I_2 = \begin{bmatrix} 1 & 0 \\ 0 & 1 \end{bmatrix}, \quad R = 1$$

Wyznaczone wartości wzmocnień $K$ regulatora LQR:
\begin{equation}
  K = \begin{bmatrix}
    9.49 & 3.16
  \end{bmatrix}
\end{equation}

Prawo sterowania dla układu nieliniowego ma postać:
$$u = -K(x - x_0) + u_0 = -9.49(\theta - \pi) - 3.16\dot{\theta}$$

\begin{figure}[H]
  \centering
  \includegraphics[width=0.75\textwidth]{lqr-wahadlo-warunki.png}
  \caption{Przebiegi pierwszej zmiennej stanu (kąt $\theta$) układu zamkniętego z regulatorem LQR
    dla różnych warunków początkowych. Regulator stabilizuje wahadło w pozycji górnej ($\theta = \pi$)
    tylko dla małych odchyleń początkowych.}
  \label{fig:lqr_wahadlo_warunki}
\end{figure}

\textbf{Czy wykorzystane rozwiązanie jest globalne?}

Nie, rozwiązanie jest lokalne -- regulator LQR zaprojektowany na podstawie liniowej aproksymacji
wokół punktu $(\pi, 0)$ gwarantuje stabilność tylko w pewnym obszarze przyciągania wokół punktu równowagi.

\textbf{Analiza wyników symulacji:}

\begin{itemize}
  \item Małe odchylenia ($\theta_0 = \pi - 0.1$, $\pi - 0.3$, $\pi - 0.5$):
        Regulator skutecznie stabilizuje wahadło w pozycji odwróconej.
        Układ dąży asymptotycznie do punktu równowagi: $\theta \to \pi$, $\dot{\theta} \to 0$.
        Linearyzacja jest wystarczająco dokładna w tym obszarze.

  \item Duże odchylenia ($\theta_0 = \pi/2$, $\theta_0 = 0$):
        Regulator traci skuteczność. Wahadło pada w dół i ustala się w pozycji dolnej
        ($\theta \approx 0$) lub wykonuje oscylacje. Nieliniowość układu dominuje, a model liniowy
        nie opisuje już poprawnie rzeczywistej dynamiki.
\end{itemize}

\textbf{Przyczyny lokalności rozwiązania:}

\begin{enumerate}
  \item Lokalność linearyzacji: Model liniowy jest dokładny tylko w małym obszarze
        wokół punktu $(\pi, 0)$. Im większe odchylenie, tym większy błąd aproksymacji.

  \item Ograniczony obszar przyciągania: Dla $|\theta - \pi| > \sim 0.5$ rad
        (około $30°$) regulator przestaje działać poprawnie.

  \item Natura problemu: Stabilizacja wahadła odwróconego to trudne zadanie sterowania.
        Globalnie stabilne rozwiązanie wymaga nieliniowych metod (np. sterowanie ślizgowe,
        backstepping, sterowanie adaptacyjne).
\end{enumerate}

\textbf{Wnioski:}

Regulator LQR dla nieliniowego wahadła zapewnia jedynie lokalne przybliżenie optymalnego
sterowania. Jest skuteczny w okolicy punktu równowagi, ale nie gwarantuje stabilności globalnej.
To typowe ograniczenie metod opartych na linearyzacji -- efektywność tylko w ograniczonym
obszarze przestrzeni stanów.

\subsection{Wykres z zadania 3.4}

Zbadano manipulator z regulatorem LQR:
\begin{itemize}
  \item regulator z nieskończonym horyzontem (ARE),
  \item regulator ze skończonym horyzontem (równanie Riccatiego rozwiązywane wstecz na przedziale $[0, t_1]$).
\end{itemize}
Parametry: $J_1=J_2=0.5$, $k=100$, $c=1$, wagi $Q=I_4$, $R=1$, warunki początkowe $x(0) = [\pi,\ 0,\ \pi/2,\ 0]^T$.

\begin{figure}[H]
  \centering
  \includegraphics[width=0.75\textwidth]{lqr-manip-5s.png}
  \caption{Manipulator, $t_1=5$ s: porównanie pozycji ogniwa $x_1$ i silnika $x_3$ dla regulatora LQR z horyzontem nieskończonym i skończonym.}
  \label{fig:lqr_manip_5s}
\end{figure}

\begin{figure}[H]
  \centering
  \includegraphics[width=0.75\textwidth]{lqr-manip-15s.png}
  \caption{Manipulator, $t_1=1.5$ s: skrócony horyzont pogarsza zbieżność dla regulatora skończonego; LQR nieskończony nadal dąży do zera.}
  \label{fig:lqr_manip_15s}
\end{figure}

\begin{figure}[H]
  \centering
  \includegraphics[width=0.75\textwidth]{lqr-manip-S-sweep.png}
  \caption{Wpływ macierzy końcowej $S$ dla horyzontu $t_1=1.5$ s: zwiększenie $S$ poprawia szybkość zbieżności regulatora skończonego (najlepiej $S=100I$); dla $x_1$ przebiegi z $S=10I$ i $100I$ niemal pokrywają się z regulatorem nieskończonym, $x_3$ poprawia się tylko nieznacznie.}
  \label{fig:lqr_manip_S}
\end{figure}

\textbf{Odpowiedzi na pytania:}

\begin{enumerate}
  \item \textbf{Czy zadanie stabilizacji układu realizowane jest poprawnie dla obu regulatorów LQR?}
        Dla $t_1=5$ s tak: oba regulatory stabilizują układ, a przebiegi są praktycznie zbieżne.

  \item \textbf{Czy zmienne stanu zbiegają do zera po skróceniu horyzontu do $t_1=1.5$ s?}
        W horyzoncie $1.5$ s \emph{żaden} regulator nie osiąga dokładnie zera: LQR nieskończony zmniejsza stany, ale w zadanym czasie zostaje spory uchyb; LQR skończony z $t_1=1.5$ s zostawia większy uchyb na $x_1$ i $x_3$.

  \item \textbf{Czy zmiana macierzy $S$ może usprawnić regulator skończony? Jaka wartość $S$ działa lepiej?}
        Tak. Większe $S$ wzmacnia karę końcową i przyspiesza zbieżność. Dla pozycji ogniwa $x_1$ przebiegi z $S=10I$ i $S=100I$ są praktycznie takie same jak dla regulatora nieskończonego; dla pozycji silnika $x_3$ widać niewielą poprawę przy większym $S$. $S=100I$ jest najbliżej rozwiązania nieskończonego, $S=10I$ daje umiarkowaną poprawę względem $S=I$, a $S=0.1I$ jest zbyt słabe.
\end{enumerate}

\end{document}
