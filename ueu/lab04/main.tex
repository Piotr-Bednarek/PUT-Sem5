\documentclass[11pt]{article}

\pdfobjcompresslevel=3    % compress PDF objects
\pdfcompresslevel=9       % compress streams more (0..9)
\pdfminorversion=4


\usepackage[utf8]{inputenc} % remove if using XeLaTeX or LuaLaTeX
\usepackage[T1]{fontenc}

% Layout & graphics
\usepackage[a4paper, total={7.5in, 9in}]{geometry}
\usepackage{graphicx}              % images
\usepackage[dvipsnames,table]{xcolor} % single xcolor invocation (keeps table option)

% Math & symbols
\usepackage{amsmath}
\usepackage{amssymb}
\usepackage{gensymb}

% Typography & layout helpers
\usepackage{microtype}
\usepackage{float}                 % [H] placement
\usepackage{caption}
\usepackage{subcaption}            % modern subfigure support (do NOT load subfig)
\usepackage{multirow}
\usepackage{titlesec}

\usepackage{lmodern}
\usepackage{microtype}

\usepackage{pgfplots}
\usepackage{siunitx}
\pgfplotsset{compat=1.18}

\definecolor{xppblue}{RGB}{37, 150, 190}

\begin{document}

\begin{table}[h]
    \centering
    \scalebox{1.5}{
        \begin{tabular}{|c|p{3cm}|l|l|cl|}
            \hline
            \multicolumn{4}{|c|}{SPRAWOZDANIE Z LABORATORIUM}                                    & \multicolumn{2}{c|}{\multirow{2}{*}{\begin{tabular}[c]{@{}c@{}}rok akademicki:\\ 2025/26\end{tabular}}}                                                          \\ \cline{1-4}
            \multicolumn{4}{|c|}{\textbf{Układy elektroniki użytkowej}}                          & \multicolumn{2}{c|}{}                                                                                                                                            \\ \hline
            \multicolumn{4}{|c|}{\multirow{2}{*}{\textit{Wzmacniacz całkujący i różniczkujący}}} & \multicolumn{2}{c|}{\multirow{2}{*}{czwartek 8:00}}                                                                                                              \\
            \multicolumn{4}{|c|}{}                                                               & \multicolumn{2}{c|}{}                                                                                                                                            \\ \hline
            \multicolumn{1}{|c|}{WARiE, AiR, sem 5}                                              & \multicolumn{3}{p{5cm}|}{\begin{tabular}[c]{@{}l@{}}1. Piotr Cybal\\ \underline{2. Piotr Bednarek}\end{tabular}} & \multicolumn{2}{l|}{\multirow{2}{*}{Punkty:}} \\ \cline{1-1}
            \multicolumn{1}{|c|}{27.10.2025}                                                     & \multicolumn{3}{c|}{}                                                                                            & \multicolumn{2}{l|}{}                         \\ \hline
        \end{tabular}
    }
\end{table}

\vspace{1\baselineskip}

\section{Sprzęt}
\begin{itemize}
    \item Numer stanowiska: 8
    \item Numer Elvisa: E
    \item Elementy elektroniczne: 2
\end{itemize}

\section{Ćwiczenie}

Podczas ćwiczenia zbudowaliśmy układy wzmacniacza całkujacego oraz różniczkującego, a
następnie zbadaliśmy ich działanie dla różnych sygnałów wejściowych dla wybranych
częstotliwości. Zbadaliśmy również wzmocnienia napięcia sinusoidalnego dla określonego
zakresu częstotliwości.

\begin{figure}[H]
    \centering
    \includegraphics[width=0.4\textwidth]{img/uklad.jpeg}
    \caption{Układ połączony na stanowisku laboratoryjnym}
\end{figure}

\newpage
Pomiary kondensatorów:

\begin{figure}[H]
    \centering
    \begin{subfigure}[b]{0.4\textwidth}
        \centering
        \includegraphics[width=\linewidth]{img/kondensator.png}
        \caption{1. kondensator}
    \end{subfigure}
    \begin{subfigure}[b]{0.4\textwidth}
        \centering
        \includegraphics[width=\linewidth]{img/kondensator2.png}
        \caption{2. kondensator}
    \end{subfigure}
    \caption{Pomiary kondensatorów}
\end{figure}

\section{Wyniki pomiarów}
\subsection{Wzmacniacz całkujący}
%dodajym zdjecie ukladu
\begin{figure}[H]
    \centering
    \includegraphics[width=0.4\textwidth]{img/uklad_calkujacy.png}
    \caption{Układ wzmacniacza całkującego}
\end{figure}
\vspace{2\baselineskip}
Wartości elementów:
\begin{itemize}
    \item $R_1 = \SI{10}{\kilo\ohm}$
    \item $C = \SI{22}{\nano\farad}$
    \item $R_2 = \SI{100}{\kilo\ohm}$
\end{itemize}
\subsubsection{Całkowanie napięcia prostokątnego - 100 Hz}
\begin{figure}[H]
    \centering
    \includegraphics[width=0.4\textwidth]{img/zad11.png}
    \caption{Sygnał wejściowy - 100 Hz}
\end{figure}

\subsubsection{Całkowanie napięcia prostokątnego - 1000 Hz}
\begin{figure}[H]
    \centering
    \includegraphics[width=0.4\textwidth]{img/zad12.png}
    \caption{Sygnał wejściowy - 1000 Hz}
\end{figure}

\subsubsection{Całkowanie napięcia prostokątnego - 10 kHz}
\begin{figure}[H]
    \centering
    \includegraphics[width=0.4\textwidth]{img/zad13.png}
    \caption{Sygnał wejściowy - 10 kHz}
\end{figure}
Brak rezystora w obwodzie sprzężenia zwrotnego powoduje nasycenie się sygnału wyjściowego
co można zobaczyć poprzez wypłaszczenie się sygnału wyjściowego na oscyloskopie.

\begin{figure}[H]
    \centering
    \includegraphics[width=0.4\textwidth]{img/bez_rezystora.png}
    \caption{Sygnał wejściowy - 10 kHz}
\end{figure}


\subsubsection{Charakterystyka częstotliwościowa wzmacniacza całkującego}

\begin{figure}[H]
    \centering
    \begin{tikzpicture}
        \begin{axis}[
            width=14cm,
            height=5cm,
            xlabel={Częstotliwość [Hz]},
            ylabel={Wzmocnienie [dB]},
            title={Zależność wzmocnienia od częstotliwości},
            grid=both,
            xmode=linear,
            scaled x ticks=false,
            legend style={
                    at={(0.5,-0.2)},
                    anchor=north,
                    legend columns=-1,
                    /tikz/every even column/.append style={column sep=0.5cm}
                },
            ymin=0,
            ymajorgrids=true,
            xtick={0,2000,4000,6000,8000,10000},
            xticklabels={0,2k,4k,6k,8k,10k},
            ]

            \addplot[
                thick,
                color=blue,
                mark=*,
            ] table[x=Freq, y=Gain, col sep=comma] {data/pomiar11.csv};
        \end{axis}
    \end{tikzpicture}
\end{figure}

\vspace{-1cm}

\begin{figure}[H]
    \centering
    \begin{tikzpicture}
        \begin{axis}[
            width=14cm,
            height=5cm,
            xlabel={Częstotliwość [Hz]},
            ylabel={Faza [°]},
            title={Zależność fazy od częstotliwości},
            grid=both,
            xmode=linear,
            scaled x ticks=false,
            legend style={
                    at={(0.5,-0.2)},
                    anchor=north,
                    legend columns=-1,
                    /tikz/every even column/.append style={column sep=0.5cm}
                },
            ymin=85,
            ymax=130,
            ymajorgrids=true,
            xtick={0,2000,4000,6000,8000,10000},
            xticklabels={0,2k,4k,6k,8k,10k},
            ]

            \addplot[
                thick,
                color=blue,
                mark=*,
            ] table[x=Freq, y=Phase, col sep=comma] {data/pomiar11.csv};
        \end{axis}
    \end{tikzpicture}
\end{figure}

\vspace{-1cm}

\begin{figure}[H]
    \centering
    \begin{tikzpicture}
        \begin{semilogxaxis}[
            width=14cm,
            height=5cm,
            xlabel={Częstotliwość [Hz]},
            ylabel={Wzmocnienie [dB]},
            title={Zależność wzmocnienia od częstotliwości (skala logarytmiczna)},
            grid=both,
            scaled x ticks=false,
            legend style={
                    at={(0.5,-0.2)},
                    anchor=north,
                    legend columns=-1,
                    /tikz/every even column/.append style={column sep=0.5cm}
                },
            ymajorgrids=true,
            ]

            \addplot[
                thick,
                color=red,
                mark=square,
            ] table[x=Freq, y=Gain, col sep=comma] {data/pomiar12.csv};
        \end{semilogxaxis}
    \end{tikzpicture}
\end{figure}

\vspace{-1cm}

\begin{figure}[H]
    \centering
    \begin{tikzpicture}
        \begin{semilogxaxis}[
            width=14cm,
            height=5cm,
            xlabel={Częstotliwość [Hz]},
            ylabel={Faza [°]},
            title={Zależność fazy od częstotliwości (skala logarytmiczna)},
            grid=both,
            scaled x ticks=false,
            legend style={
                    at={(0.5,-0.2)},
                    anchor=north,
                    legend columns=-1,
                    /tikz/every even column/.append style={column sep=0.5cm}
                },
            ymin=85,
            ymax=130,
            ymajorgrids=true,
            ]

            \addplot[
                thick,
                color=red,
                mark=square,
            ] table[x=Freq, y=Phase, col sep=comma] {data/pomiar12.csv};
        \end{semilogxaxis}
    \end{tikzpicture}
\end{figure}

\subsection{Wzmacniacz różniczkujący}

\begin{figure}[H]
    \centering
    \includegraphics[width=0.4\textwidth]{img/uklad_rozniczkujacy.png}
    \caption{Układ wzmacniacza różniczkującego}
\end{figure}

Wartości elementów:
\begin{itemize}
    \item $R_1 = \SI{11}{\kilo\ohm}$
    \item $C = \SI{22}{\nano\farad}$
    \item $R_2 = \SI{10}{\kilo\ohm}$
\end{itemize}

\subsubsection{Różniczkowanie napięcia trójkątnego - 100 Hz}
\begin{figure}[H]
    \centering
    \includegraphics[width=0.4\textwidth]{img/zad21.png}
    \caption{Sygnał wejściowy - 100 Hz}
\end{figure}


\subsubsection{Różniczkowanie napięcia trójkątnego - 1000 Hz}
\begin{figure}[H]
    \centering
    \includegraphics[width=0.4\textwidth]{img/zad22.png}
    \caption{Sygnał wejściowy - 1000 Hz}
\end{figure}

\subsubsection{Różniczkowanie napięcia trójkątnego - 10 kHz}
\begin{figure}[H]
    \centering
    \includegraphics[width=0.4\textwidth]{img/zad23.png}
    \caption{Sygnał wejściowy - 10 kHz}
\end{figure}


\subsubsection{ Charakterystyka częstotliwościowa wzmacniacza różniczkującego}

\begin{figure}[H]
    \centering
    \begin{tikzpicture}
        \begin{axis}[
            width=14cm,
            height=5cm,
            xlabel={Częstotliwość [Hz]},
            ylabel={Wzmocnienie [dB]},
            title={Zależność wzmocnienia od częstotliwości},
            grid=both,
            xmode=linear,
            scaled x ticks=false,
            legend style={
                    at={(0.5,-0.2)},
                    anchor=north,
                    legend columns=-1,
                    /tikz/every even column/.append style={column sep=0.5cm}
                },
            ymin=0,
            ymajorgrids=true,
            xtick={0,2000,4000,6000,8000,10000},
            xticklabels={0,2k,4k,6k,8k,10k},
            ]

            \addplot[
                thick,
                color=blue,
                mark=*,
            ] table[x=Freq, y=Gain, col sep=comma] {data/pomiar21.csv};
        \end{axis}
    \end{tikzpicture}
\end{figure}

\vspace{-1cm}

\begin{figure}[H]
    \centering
    \begin{tikzpicture}
        \begin{axis}[
            width=14cm,
            height=5cm,
            xlabel={Częstotliwość [Hz]},
            ylabel={Faza [°]},
            title={Zależność fazy od częstotliwości},
            grid=both,
            xmode=linear,
            scaled x ticks=false,
            legend style={
                    at={(0.5,-0.2)},
                    anchor=north,
                    legend columns=-1,
                    /tikz/every even column/.append style={column sep=0.5cm}
                },
            ymajorgrids=true,
            xtick={0,2000,4000,6000,8000,10000},
            xticklabels={0,2k,4k,6k,8k,10k},
            ]

            \addplot[
                thick,
                color=blue,
                mark=*,
            ] table[x=Freq, y=Phase, col sep=comma] {data/pomiar21.csv};
        \end{axis}
    \end{tikzpicture}
\end{figure}

\vspace{-1cm}

\begin{figure}[H]
    \centering
    \begin{tikzpicture}
        \begin{semilogxaxis}[
            width=14cm,
            height=5cm,
            xlabel={Częstotliwość [Hz]},
            ylabel={Wzmocnienie [dB]},
            title={Zależność wzmocnienia od częstotliwości (skala logarytmiczna)},
            grid=both,
            scaled x ticks=false,
            legend style={
                    at={(0.5,-0.2)},
                    anchor=north,
                    legend columns=-1,
                    /tikz/every even column/.append style={column sep=0.5cm}
                },
            ymajorgrids=true,
            ]

            \addplot[
                thick,
                color=red,
                mark=square,
            ] table[x=Freq, y=Gain, col sep=comma] {data/pomiar22.csv};
        \end{semilogxaxis}
    \end{tikzpicture}
\end{figure}

\vspace{-1cm}

\begin{figure}[H]
    \centering
    \begin{tikzpicture}
        \begin{semilogxaxis}[
            width=14cm,
            height=5cm,
            xlabel={Częstotliwość [Hz]},
            ylabel={Faza [°]},
            title={Zależność fazy od częstotliwości (skala logarytmiczna)},
            grid=both,
            scaled x ticks=false,
            legend style={
                    at={(0.5,-0.2)},
                    anchor=north,
                    legend columns=-1,
                    /tikz/every even column/.append style={column sep=0.5cm}
                },
            ymajorgrids=true,
            ]

            \addplot[
                thick,
                color=red,
                mark=square,
            ] table[x=Freq, y=Phase, col sep=comma] {data/pomiar22.csv};
        \end{semilogxaxis}
    \end{tikzpicture}
\end{figure}

\section{Jak działa Bode Analyzer?}
Funckja Bode Analyzer polega na automatycznym testowaniu danego układu dla szerokiego
zakresu zadanej częstotliwości. Robi to podając na wejście sygnał o zmiennej
częstotliwości przy czym mierzy wzmocnienie oraz fazę sygnału wyjściowego w stosunku do
sygnału wejściowego. Wyniki te są następnie przedstawiane na wykresach w wybranej skali.
\section{Obserwacje}
\subsection{Układ całkujący}
Układ poprawnie całkuje sygnały prostokątne do sygnałów trójkątnych. Charaktrystyka
częstotliwościowa układu pokazuje, że wzmocnienie układu maleje wraz ze wzrostem
częstotliwości sygnału wejściowego co jest typowe dla filtru dolnoprzepustowego.
Pomiar z odłączonym rezystorem w obwodzie sprzężenia zwrotnego pokazuje, że sygnał wyjściowy
ulega nasyceniu.
\subsection{Układ różniczkujący}
Układ zgodnie z teorią różniczkuje sygnały trójkątne do sygnałów prostokątnych. Charaktrystyka
częstotliwościowa układu pokazuje, że wzmocnienie układu rośnie wraz ze wzrostem
częstotliwości sygnału wejściowego co jest typowe dla filtru górnoprzepustowego.
Przy wyższych częstotliwościach sygnału wejściowego sygnał wyjściowy zaczyna być zniekształcony
ze względu na ograniczenia pasma przenoszenia wzmacniacza operacyjnego.


\section{Wnioski}
Zbudowane układy wzmacniacza całkującego oraz różniczkującego działają zgodnie z teorią.
Charakterystyki częstotliwościowe układów potwierdzają ich działanie jako filtru
dolnoprzepustowego oraz górnoprzepustowego odpowiednio. W praktye jednak układy te
są ograniczone przez pasmo przenoszenia wzmacniacza operacyjnego, co powoduje zniekształcenia
sygnałów przy wyższych częstotliwościach.
Praktyczne ograniczenia układów:
\begin{itemize}
    \item Układ całkujący bez rezystora w obwodzie sprzężenia zwrotnego prowadzi do
          nasycenia się sygnału wyjściowego
    \item Pomiary sygnałów wyjściowych dla układu różniczkującego podkreśliły podatność na szumy,
          zwłaszcza przy niskich częstotliwościach.
\end{itemize}


\end{document}
