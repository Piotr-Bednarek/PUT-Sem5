\documentclass[11pt]{article}

\pdfobjcompresslevel=3    % compress PDF objects
\pdfcompresslevel=9       % compress streams more (0..9)
\pdfminorversion=4


\usepackage[utf8]{inputenc} % remove if using XeLaTeX or LuaLaTeX
\usepackage[T1]{fontenc}
\usepackage[polish]{babel}

% Layout & graphics
\usepackage[a4paper, total={7.5in, 9in}]{geometry}
\usepackage{graphicx}              % images
\usepackage[dvipsnames,table]{xcolor} % single xcolor invocation (keeps table option)

% Math & symbols
\usepackage{amsmath}
\usepackage{amssymb}
\usepackage{gensymb}

% Typography & layout helpers
\usepackage{float}                 % [H] placement
\usepackage{caption}
\usepackage{subcaption}            % modern subfigure support (do NOT load subfig)
\usepackage{multirow}
\usepackage{titlesec}

\usepackage{lmodern}

\usepackage{pgfplots}
\usepackage{siunitx}
\pgfplotsset{compat=1.18}

\definecolor{xppblue}{RGB}{37, 150, 190}

\title{Ograniczniki napięcia}
\author{Piotr Bednarek}
\date{}

\begin{document}

\maketitle
\section{Układy pomiarowe}
\begin{figure}[H]
    \centering
    \includegraphics[width=0.5\textwidth]{uklad.png}
    \caption{Ukady pomiarowe "soft" i "hard" limiting}
    \label{fig:circuit1}
\end{figure}

\section{Sygnały}
\subsection{Poniżej progu ograniczania}
\begin{figure}[H]
    \centering
    \includegraphics[width=0.75\textwidth]{1.png}
    \caption{Wejścia i wyjścia obu układów}
    \label{fig:1}
\end{figure}

Dla sygnałów poniżej progu ograniczania oba układy zachowują się podobnie, co widać na
rysunku \ref{fig:1}. Wyjścia obu układów są niemal identyczne z wejściem, co oznacza, że
sygnał nie został ograniczony.

\subsection{Powyżej progu ograniczania}
\begin{figure}[H]
    \centering
    \includegraphics[width=0.75\textwidth]{12.png}
    \caption{Wejścia i wyjścia obu układów}
    \label{fig:2}
\end{figure}

W przypadku sygnałów przekraczających próg ograniczania, oba układy wykazują ograniczenia
sygnału wyjściowego w różny sposób.

\subsection{Działanie progu w zależności od wzmocnienia wzmacniacza operacyjnego}
\begin{figure}[H]
    \centering
    \includegraphics[width=0.75\textwidth]{2.png}
    \caption{Kilka wariantów wzmocnień sygnału wejściowego}
    \label{fig:3}
\end{figure}

Próg zadziałania ogranicznika zależy od napięcia Zenera. Zmiana
wzmocnienia wpływa na wartość napięcia wyjściowego, co jest widoczne na rysunku
\ref{fig:3}, lecz nie wpływa na sam próg ograniczania.

\subsection{Ograniczenie amplitudy przez napięcia zasilające}
\begin{figure}[H]
    \centering
    \includegraphics[width=0.75\textwidth]{3.png}
    \caption{Kilka wariantów wzmocnień sygnału wejściowego}
    \label{fig:33}
\end{figure}

Przy dużej amplitudzie sygnału wyjście wzmacniacza osiąga granice napięć zasilających, co
prowadzi do obcięcia (saturacji) sygnału.

\section{Zniekształcenia harmoniczne wprowadzane przez ogranicznik}


\subsection{Widmo sygnału wyjściowego}

\subsubsection{Widmo sygnału wyjściowego dla obu wariantów układu}
\begin{figure}[H]
    \centering
    \includegraphics[width=0.75\textwidth]{widmo1.png}
    \caption{Widmo sygnału wyjściowego dla "soft" \ limiting}
    \label{fig:widmo1}
\end{figure}
\begin{figure}[H]
    \centering
    \includegraphics[width=0.75\textwidth]{widmo2.png}
    \caption{Widmo sygnału wyjściowego dla "hard" \ limiting}
    \label{fig:widmo2}
\end{figure}


\subsubsection{Wartość parametru THD}
Na podstawie analizy widma odczytano z LTSpice wartość współczynnika zniekształceń
harmonicznych THD (Total Harmonic Distortion).
\begin{itemize}
    \item Dla ograniczenia miękkiego (soft limiting) THD $\approx$ 20.87\%
    \item Dla ograniczenia ostrego (hard limiting) THD $\approx$ 20.73\%
\end{itemize}

\subsubsection{Harmoniczne obecne w sygnale wyjściowym}
W widmie sygnału wyjściowego pojawiają się głównie harmoniczne nieparzyste (1 kHz, 3 kHz, 5 kHz, 7 kHz),
charakterystyczne dla nieliniowości symetrycznych.

\subsubsection{Zastosowanie parametru THD}
Parametr THD służy do oceny jakości przetwarzania sygnałów w układach elektronicznych.
Stosuje się go m.in. do charakterystyki wzmacniaczy audio, przetworników
analogowo-cyfrowych oraz układów ograniczających, gdzie informuje o stopniu
zniekształcenia sygnału wyjściowego względem idealnego przebiegu sinusoidalnego.


\end{document}
